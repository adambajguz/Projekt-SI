
% Default to the notebook output style

    


% Inherit from the specified cell style.




    
\documentclass[landscape, a4paper, 10pt]{article}

    
    
    \usepackage[T1]{fontenc}
    % Nicer default font (+ math font) than Computer Modern for most use cases
    \usepackage{mathpazo}

    % Basic figure setup, for now with no caption control since it's done
    % automatically by Pandoc (which extracts ![](path) syntax from Markdown).
    \usepackage{graphicx}
    % We will generate all images so they have a width \maxwidth. This means
    % that they will get their normal width if they fit onto the page, but
    % are scaled down if they would overflow the margins.
    \makeatletter
    \def\maxwidth{\ifdim\Gin@nat@width>\linewidth\linewidth
    \else\Gin@nat@width\fi}
    \makeatother
    \let\Oldincludegraphics\includegraphics
    % Set max figure width to be 80% of text width, for now hardcoded.
    \renewcommand{\includegraphics}[1]{\Oldincludegraphics[width=.8\maxwidth]{#1}}
    % Ensure that by default, figures have no caption (until we provide a
    % proper Figure object with a Caption API and a way to capture that
    % in the conversion process - todo).
    \usepackage{caption}
    \DeclareCaptionLabelFormat{nolabel}{}
    \captionsetup{labelformat=nolabel}

    \usepackage{adjustbox} % Used to constrain images to a maximum size 
    \usepackage{xcolor} % Allow colors to be defined
    \usepackage{enumerate} % Needed for markdown enumerations to work
    \usepackage{geometry} % Used to adjust the document margins
    \usepackage{amsmath} % Equations
    \usepackage{amssymb} % Equations
    \usepackage{textcomp} % defines textquotesingle
    % Hack from http://tex.stackexchange.com/a/47451/13684:
    \AtBeginDocument{%
        \def\PYZsq{\textquotesingle}% Upright quotes in Pygmentized code
    }
    \usepackage{upquote} % Upright quotes for verbatim code
    \usepackage{eurosym} % defines \euro
    \usepackage[mathletters]{ucs} % Extended unicode (utf-8) support
    \usepackage[utf8x]{inputenc} % Allow utf-8 characters in the tex document
    \usepackage{fancyvrb} % verbatim replacement that allows latex
    \usepackage{grffile} % extends the file name processing of package graphics 
                         % to support a larger range 
    % The hyperref package gives us a pdf with properly built
    % internal navigation ('pdf bookmarks' for the table of contents,
    % internal cross-reference links, web links for URLs, etc.)
    \usepackage{hyperref}
    \usepackage{longtable} % longtable support required by pandoc >1.10
    \usepackage{booktabs}  % table support for pandoc > 1.12.2
    \usepackage[inline]{enumitem} % IRkernel/repr support (it uses the enumerate* environment)
    \usepackage[normalem]{ulem} % ulem is needed to support strikethroughs (\sout)
                                % normalem makes italics be italics, not underlines
    

    
    
    % Colors for the hyperref package
    \definecolor{urlcolor}{rgb}{0,.145,.698}
    \definecolor{linkcolor}{rgb}{.71,0.21,0.01}
    \definecolor{citecolor}{rgb}{.12,.54,.11}

    % ANSI colors
    \definecolor{ansi-black}{HTML}{3E424D}
    \definecolor{ansi-black-intense}{HTML}{282C36}
    \definecolor{ansi-red}{HTML}{E75C58}
    \definecolor{ansi-red-intense}{HTML}{B22B31}
    \definecolor{ansi-green}{HTML}{00A250}
    \definecolor{ansi-green-intense}{HTML}{007427}
    \definecolor{ansi-yellow}{HTML}{DDB62B}
    \definecolor{ansi-yellow-intense}{HTML}{B27D12}
    \definecolor{ansi-blue}{HTML}{208FFB}
    \definecolor{ansi-blue-intense}{HTML}{0065CA}
    \definecolor{ansi-magenta}{HTML}{D160C4}
    \definecolor{ansi-magenta-intense}{HTML}{A03196}
    \definecolor{ansi-cyan}{HTML}{60C6C8}
    \definecolor{ansi-cyan-intense}{HTML}{258F8F}
    \definecolor{ansi-white}{HTML}{C5C1B4}
    \definecolor{ansi-white-intense}{HTML}{A1A6B2}

    % commands and environments needed by pandoc snippets
    % extracted from the output of `pandoc -s`
    \providecommand{\tightlist}{%
      \setlength{\itemsep}{0pt}\setlength{\parskip}{0pt}}
    \DefineVerbatimEnvironment{Highlighting}{Verbatim}{commandchars=\\\{\}}
    % Add ',fontsize=\small' for more characters per line
    \newenvironment{Shaded}{}{}
    \newcommand{\KeywordTok}[1]{\textcolor[rgb]{0.00,0.44,0.13}{\textbf{{#1}}}}
    \newcommand{\DataTypeTok}[1]{\textcolor[rgb]{0.56,0.13,0.00}{{#1}}}
    \newcommand{\DecValTok}[1]{\textcolor[rgb]{0.25,0.63,0.44}{{#1}}}
    \newcommand{\BaseNTok}[1]{\textcolor[rgb]{0.25,0.63,0.44}{{#1}}}
    \newcommand{\FloatTok}[1]{\textcolor[rgb]{0.25,0.63,0.44}{{#1}}}
    \newcommand{\CharTok}[1]{\textcolor[rgb]{0.25,0.44,0.63}{{#1}}}
    \newcommand{\StringTok}[1]{\textcolor[rgb]{0.25,0.44,0.63}{{#1}}}
    \newcommand{\CommentTok}[1]{\textcolor[rgb]{0.38,0.63,0.69}{\textit{{#1}}}}
    \newcommand{\OtherTok}[1]{\textcolor[rgb]{0.00,0.44,0.13}{{#1}}}
    \newcommand{\AlertTok}[1]{\textcolor[rgb]{1.00,0.00,0.00}{\textbf{{#1}}}}
    \newcommand{\FunctionTok}[1]{\textcolor[rgb]{0.02,0.16,0.49}{{#1}}}
    \newcommand{\RegionMarkerTok}[1]{{#1}}
    \newcommand{\ErrorTok}[1]{\textcolor[rgb]{1.00,0.00,0.00}{\textbf{{#1}}}}
    \newcommand{\NormalTok}[1]{{#1}}
    
    % Additional commands for more recent versions of Pandoc
    \newcommand{\ConstantTok}[1]{\textcolor[rgb]{0.53,0.00,0.00}{{#1}}}
    \newcommand{\SpecialCharTok}[1]{\textcolor[rgb]{0.25,0.44,0.63}{{#1}}}
    \newcommand{\VerbatimStringTok}[1]{\textcolor[rgb]{0.25,0.44,0.63}{{#1}}}
    \newcommand{\SpecialStringTok}[1]{\textcolor[rgb]{0.73,0.40,0.53}{{#1}}}
    \newcommand{\ImportTok}[1]{{#1}}
    \newcommand{\DocumentationTok}[1]{\textcolor[rgb]{0.73,0.13,0.13}{\textit{{#1}}}}
    \newcommand{\AnnotationTok}[1]{\textcolor[rgb]{0.38,0.63,0.69}{\textbf{\textit{{#1}}}}}
    \newcommand{\CommentVarTok}[1]{\textcolor[rgb]{0.38,0.63,0.69}{\textbf{\textit{{#1}}}}}
    \newcommand{\VariableTok}[1]{\textcolor[rgb]{0.10,0.09,0.49}{{#1}}}
    \newcommand{\ControlFlowTok}[1]{\textcolor[rgb]{0.00,0.44,0.13}{\textbf{{#1}}}}
    \newcommand{\OperatorTok}[1]{\textcolor[rgb]{0.40,0.40,0.40}{{#1}}}
    \newcommand{\BuiltInTok}[1]{{#1}}
    \newcommand{\ExtensionTok}[1]{{#1}}
    \newcommand{\PreprocessorTok}[1]{\textcolor[rgb]{0.74,0.48,0.00}{{#1}}}
    \newcommand{\AttributeTok}[1]{\textcolor[rgb]{0.49,0.56,0.16}{{#1}}}
    \newcommand{\InformationTok}[1]{\textcolor[rgb]{0.38,0.63,0.69}{\textbf{\textit{{#1}}}}}
    \newcommand{\WarningTok}[1]{\textcolor[rgb]{0.38,0.63,0.69}{\textbf{\textit{{#1}}}}}
    
    
    % Define a nice break command that doesn't care if a line doesn't already
    % exist.
    \def\br{\hspace*{\fill} \\* }
    % Math Jax compatability definitions
    \def\gt{>}
    \def\lt{<}
    % Document parameters
    \title{ForestFires}
    
    
    

    % Pygments definitions
    
\makeatletter
\def\PY@reset{\let\PY@it=\relax \let\PY@bf=\relax%
    \let\PY@ul=\relax \let\PY@tc=\relax%
    \let\PY@bc=\relax \let\PY@ff=\relax}
\def\PY@tok#1{\csname PY@tok@#1\endcsname}
\def\PY@toks#1+{\ifx\relax#1\empty\else%
    \PY@tok{#1}\expandafter\PY@toks\fi}
\def\PY@do#1{\PY@bc{\PY@tc{\PY@ul{%
    \PY@it{\PY@bf{\PY@ff{#1}}}}}}}
\def\PY#1#2{\PY@reset\PY@toks#1+\relax+\PY@do{#2}}

\expandafter\def\csname PY@tok@mb\endcsname{\def\PY@tc##1{\textcolor[rgb]{0.40,0.40,0.40}{##1}}}
\expandafter\def\csname PY@tok@c1\endcsname{\let\PY@it=\textit\def\PY@tc##1{\textcolor[rgb]{0.25,0.50,0.50}{##1}}}
\expandafter\def\csname PY@tok@kr\endcsname{\let\PY@bf=\textbf\def\PY@tc##1{\textcolor[rgb]{0.00,0.50,0.00}{##1}}}
\expandafter\def\csname PY@tok@s1\endcsname{\def\PY@tc##1{\textcolor[rgb]{0.73,0.13,0.13}{##1}}}
\expandafter\def\csname PY@tok@nt\endcsname{\let\PY@bf=\textbf\def\PY@tc##1{\textcolor[rgb]{0.00,0.50,0.00}{##1}}}
\expandafter\def\csname PY@tok@gr\endcsname{\def\PY@tc##1{\textcolor[rgb]{1.00,0.00,0.00}{##1}}}
\expandafter\def\csname PY@tok@ge\endcsname{\let\PY@it=\textit}
\expandafter\def\csname PY@tok@gi\endcsname{\def\PY@tc##1{\textcolor[rgb]{0.00,0.63,0.00}{##1}}}
\expandafter\def\csname PY@tok@nn\endcsname{\let\PY@bf=\textbf\def\PY@tc##1{\textcolor[rgb]{0.00,0.00,1.00}{##1}}}
\expandafter\def\csname PY@tok@sc\endcsname{\def\PY@tc##1{\textcolor[rgb]{0.73,0.13,0.13}{##1}}}
\expandafter\def\csname PY@tok@vg\endcsname{\def\PY@tc##1{\textcolor[rgb]{0.10,0.09,0.49}{##1}}}
\expandafter\def\csname PY@tok@err\endcsname{\def\PY@bc##1{\setlength{\fboxsep}{0pt}\fcolorbox[rgb]{1.00,0.00,0.00}{1,1,1}{\strut ##1}}}
\expandafter\def\csname PY@tok@se\endcsname{\let\PY@bf=\textbf\def\PY@tc##1{\textcolor[rgb]{0.73,0.40,0.13}{##1}}}
\expandafter\def\csname PY@tok@vi\endcsname{\def\PY@tc##1{\textcolor[rgb]{0.10,0.09,0.49}{##1}}}
\expandafter\def\csname PY@tok@no\endcsname{\def\PY@tc##1{\textcolor[rgb]{0.53,0.00,0.00}{##1}}}
\expandafter\def\csname PY@tok@w\endcsname{\def\PY@tc##1{\textcolor[rgb]{0.73,0.73,0.73}{##1}}}
\expandafter\def\csname PY@tok@c\endcsname{\let\PY@it=\textit\def\PY@tc##1{\textcolor[rgb]{0.25,0.50,0.50}{##1}}}
\expandafter\def\csname PY@tok@o\endcsname{\def\PY@tc##1{\textcolor[rgb]{0.40,0.40,0.40}{##1}}}
\expandafter\def\csname PY@tok@nl\endcsname{\def\PY@tc##1{\textcolor[rgb]{0.63,0.63,0.00}{##1}}}
\expandafter\def\csname PY@tok@ni\endcsname{\let\PY@bf=\textbf\def\PY@tc##1{\textcolor[rgb]{0.60,0.60,0.60}{##1}}}
\expandafter\def\csname PY@tok@kn\endcsname{\let\PY@bf=\textbf\def\PY@tc##1{\textcolor[rgb]{0.00,0.50,0.00}{##1}}}
\expandafter\def\csname PY@tok@sa\endcsname{\def\PY@tc##1{\textcolor[rgb]{0.73,0.13,0.13}{##1}}}
\expandafter\def\csname PY@tok@kp\endcsname{\def\PY@tc##1{\textcolor[rgb]{0.00,0.50,0.00}{##1}}}
\expandafter\def\csname PY@tok@cs\endcsname{\let\PY@it=\textit\def\PY@tc##1{\textcolor[rgb]{0.25,0.50,0.50}{##1}}}
\expandafter\def\csname PY@tok@cm\endcsname{\let\PY@it=\textit\def\PY@tc##1{\textcolor[rgb]{0.25,0.50,0.50}{##1}}}
\expandafter\def\csname PY@tok@mo\endcsname{\def\PY@tc##1{\textcolor[rgb]{0.40,0.40,0.40}{##1}}}
\expandafter\def\csname PY@tok@mh\endcsname{\def\PY@tc##1{\textcolor[rgb]{0.40,0.40,0.40}{##1}}}
\expandafter\def\csname PY@tok@go\endcsname{\def\PY@tc##1{\textcolor[rgb]{0.53,0.53,0.53}{##1}}}
\expandafter\def\csname PY@tok@sx\endcsname{\def\PY@tc##1{\textcolor[rgb]{0.00,0.50,0.00}{##1}}}
\expandafter\def\csname PY@tok@kc\endcsname{\let\PY@bf=\textbf\def\PY@tc##1{\textcolor[rgb]{0.00,0.50,0.00}{##1}}}
\expandafter\def\csname PY@tok@gt\endcsname{\def\PY@tc##1{\textcolor[rgb]{0.00,0.27,0.87}{##1}}}
\expandafter\def\csname PY@tok@gs\endcsname{\let\PY@bf=\textbf}
\expandafter\def\csname PY@tok@fm\endcsname{\def\PY@tc##1{\textcolor[rgb]{0.00,0.00,1.00}{##1}}}
\expandafter\def\csname PY@tok@nv\endcsname{\def\PY@tc##1{\textcolor[rgb]{0.10,0.09,0.49}{##1}}}
\expandafter\def\csname PY@tok@mi\endcsname{\def\PY@tc##1{\textcolor[rgb]{0.40,0.40,0.40}{##1}}}
\expandafter\def\csname PY@tok@sd\endcsname{\let\PY@it=\textit\def\PY@tc##1{\textcolor[rgb]{0.73,0.13,0.13}{##1}}}
\expandafter\def\csname PY@tok@ch\endcsname{\let\PY@it=\textit\def\PY@tc##1{\textcolor[rgb]{0.25,0.50,0.50}{##1}}}
\expandafter\def\csname PY@tok@sr\endcsname{\def\PY@tc##1{\textcolor[rgb]{0.73,0.40,0.53}{##1}}}
\expandafter\def\csname PY@tok@nb\endcsname{\def\PY@tc##1{\textcolor[rgb]{0.00,0.50,0.00}{##1}}}
\expandafter\def\csname PY@tok@kd\endcsname{\let\PY@bf=\textbf\def\PY@tc##1{\textcolor[rgb]{0.00,0.50,0.00}{##1}}}
\expandafter\def\csname PY@tok@k\endcsname{\let\PY@bf=\textbf\def\PY@tc##1{\textcolor[rgb]{0.00,0.50,0.00}{##1}}}
\expandafter\def\csname PY@tok@sh\endcsname{\def\PY@tc##1{\textcolor[rgb]{0.73,0.13,0.13}{##1}}}
\expandafter\def\csname PY@tok@m\endcsname{\def\PY@tc##1{\textcolor[rgb]{0.40,0.40,0.40}{##1}}}
\expandafter\def\csname PY@tok@nf\endcsname{\def\PY@tc##1{\textcolor[rgb]{0.00,0.00,1.00}{##1}}}
\expandafter\def\csname PY@tok@vm\endcsname{\def\PY@tc##1{\textcolor[rgb]{0.10,0.09,0.49}{##1}}}
\expandafter\def\csname PY@tok@si\endcsname{\let\PY@bf=\textbf\def\PY@tc##1{\textcolor[rgb]{0.73,0.40,0.53}{##1}}}
\expandafter\def\csname PY@tok@nc\endcsname{\let\PY@bf=\textbf\def\PY@tc##1{\textcolor[rgb]{0.00,0.00,1.00}{##1}}}
\expandafter\def\csname PY@tok@nd\endcsname{\def\PY@tc##1{\textcolor[rgb]{0.67,0.13,1.00}{##1}}}
\expandafter\def\csname PY@tok@gh\endcsname{\let\PY@bf=\textbf\def\PY@tc##1{\textcolor[rgb]{0.00,0.00,0.50}{##1}}}
\expandafter\def\csname PY@tok@vc\endcsname{\def\PY@tc##1{\textcolor[rgb]{0.10,0.09,0.49}{##1}}}
\expandafter\def\csname PY@tok@cpf\endcsname{\let\PY@it=\textit\def\PY@tc##1{\textcolor[rgb]{0.25,0.50,0.50}{##1}}}
\expandafter\def\csname PY@tok@ss\endcsname{\def\PY@tc##1{\textcolor[rgb]{0.10,0.09,0.49}{##1}}}
\expandafter\def\csname PY@tok@bp\endcsname{\def\PY@tc##1{\textcolor[rgb]{0.00,0.50,0.00}{##1}}}
\expandafter\def\csname PY@tok@ne\endcsname{\let\PY@bf=\textbf\def\PY@tc##1{\textcolor[rgb]{0.82,0.25,0.23}{##1}}}
\expandafter\def\csname PY@tok@il\endcsname{\def\PY@tc##1{\textcolor[rgb]{0.40,0.40,0.40}{##1}}}
\expandafter\def\csname PY@tok@gp\endcsname{\let\PY@bf=\textbf\def\PY@tc##1{\textcolor[rgb]{0.00,0.00,0.50}{##1}}}
\expandafter\def\csname PY@tok@cp\endcsname{\def\PY@tc##1{\textcolor[rgb]{0.74,0.48,0.00}{##1}}}
\expandafter\def\csname PY@tok@sb\endcsname{\def\PY@tc##1{\textcolor[rgb]{0.73,0.13,0.13}{##1}}}
\expandafter\def\csname PY@tok@kt\endcsname{\def\PY@tc##1{\textcolor[rgb]{0.69,0.00,0.25}{##1}}}
\expandafter\def\csname PY@tok@na\endcsname{\def\PY@tc##1{\textcolor[rgb]{0.49,0.56,0.16}{##1}}}
\expandafter\def\csname PY@tok@dl\endcsname{\def\PY@tc##1{\textcolor[rgb]{0.73,0.13,0.13}{##1}}}
\expandafter\def\csname PY@tok@ow\endcsname{\let\PY@bf=\textbf\def\PY@tc##1{\textcolor[rgb]{0.67,0.13,1.00}{##1}}}
\expandafter\def\csname PY@tok@s\endcsname{\def\PY@tc##1{\textcolor[rgb]{0.73,0.13,0.13}{##1}}}
\expandafter\def\csname PY@tok@mf\endcsname{\def\PY@tc##1{\textcolor[rgb]{0.40,0.40,0.40}{##1}}}
\expandafter\def\csname PY@tok@s2\endcsname{\def\PY@tc##1{\textcolor[rgb]{0.73,0.13,0.13}{##1}}}
\expandafter\def\csname PY@tok@gu\endcsname{\let\PY@bf=\textbf\def\PY@tc##1{\textcolor[rgb]{0.50,0.00,0.50}{##1}}}
\expandafter\def\csname PY@tok@gd\endcsname{\def\PY@tc##1{\textcolor[rgb]{0.63,0.00,0.00}{##1}}}

\def\PYZbs{\char`\\}
\def\PYZus{\char`\_}
\def\PYZob{\char`\{}
\def\PYZcb{\char`\}}
\def\PYZca{\char`\^}
\def\PYZam{\char`\&}
\def\PYZlt{\char`\<}
\def\PYZgt{\char`\>}
\def\PYZsh{\char`\#}
\def\PYZpc{\char`\%}
\def\PYZdl{\char`\$}
\def\PYZhy{\char`\-}
\def\PYZsq{\char`\'}
\def\PYZdq{\char`\"}
\def\PYZti{\char`\~}
% for compatibility with earlier versions
\def\PYZat{@}
\def\PYZlb{[}
\def\PYZrb{]}
\makeatother


    % Exact colors from NB
    \definecolor{incolor}{rgb}{0.0, 0.0, 0.5}
    \definecolor{outcolor}{rgb}{0.545, 0.0, 0.0}



    
    % Prevent overflowing lines due to hard-to-break entities
    \sloppy 
    % Setup hyperref package
    \hypersetup{
      breaklinks=true,  % so long urls are correctly broken across lines
      colorlinks=true,
      urlcolor=urlcolor,
      linkcolor=linkcolor,
      citecolor=citecolor,
      }
    % Slightly bigger margins than the latex defaults
    
    \geometry{verbose,tmargin=1in,bmargin=1in,lmargin=1in,rmargin=1in}
    
    

    \begin{document}
    
    
\setcounter{page}{12}
    

    
    Załadowanie bibliotek

    \begin{Verbatim}[commandchars=\\\{\}]
{\color{incolor}In [{\color{incolor}1}]:} \PY{k+kn}{import} \PY{n+nn}{pandas} \PY{k}{as} \PY{n+nn}{pd}
        \PY{k+kn}{import} \PY{n+nn}{numpy} \PY{k}{as} \PY{n+nn}{np}
        \PY{k+kn}{import} \PY{n+nn}{seaborn} \PY{k}{as} \PY{n+nn}{sns}
        \PY{o}{\PYZpc{}}\PY{k}{matplotlib} inline
        \PY{k+kn}{from} \PY{n+nn}{matplotlib} \PY{k}{import} \PY{n}{pyplot} \PY{k}{as} \PY{n}{plt}
        
        \PY{c+c1}{\PYZsh{}Ustalenie stylu wykresów jako ggplot}
        \PY{c+c1}{\PYZsh{}plt.style.use(\PYZsq{}ggplot\PYZsq{})}
        \PY{k+kn}{from} \PY{n+nn}{sklearn} \PY{k}{import} \PY{n}{preprocessing}
        \PY{k+kn}{from} \PY{n+nn}{sklearn} \PY{k}{import} \PY{n}{tree}
        \PY{k+kn}{import} \PY{n+nn}{sklearn}\PY{n+nn}{.}\PY{n+nn}{metrics} \PY{k}{as} \PY{n+nn}{metrics}
        \PY{k+kn}{from} \PY{n+nn}{sklearn}\PY{n+nn}{.}\PY{n+nn}{model\PYZus{}selection} \PY{k}{import} \PY{n}{train\PYZus{}test\PYZus{}split}
        \PY{k+kn}{import} \PY{n+nn}{pydotplus}
        \PY{k+kn}{from} \PY{n+nn}{IPython}\PY{n+nn}{.}\PY{n+nn}{display} \PY{k}{import} \PY{n}{Image}  
        \PY{k+kn}{from} \PY{n+nn}{sklearn}\PY{n+nn}{.}\PY{n+nn}{model\PYZus{}selection} \PY{k}{import} \PY{n}{cross\PYZus{}val\PYZus{}score}
        \PY{k+kn}{from} \PY{n+nn}{sklearn}\PY{n+nn}{.}\PY{n+nn}{metrics} \PY{k}{import} \PY{n}{r2\PYZus{}score}\PY{p}{,} \PY{n}{mean\PYZus{}absolute\PYZus{}error}
\end{Verbatim}

    { \hspace*{\fill} \\}
    
    Wczytanie danych

    \begin{Verbatim}[commandchars=\\\{\}]
{\color{incolor}In [{\color{incolor}2}]:} \PY{k}{def} \PY{n+nf}{dataframe\PYZus{}size\PYZus{}formated}\PY{p}{(}\PY{n}{dataframe}\PY{p}{,} \PY{n}{extra}\PY{o}{=}\PY{l+s+s2}{\PYZdq{}}\PY{l+s+s2}{\PYZdq{}}\PY{p}{)}\PY{p}{:}
            \PY{n+nb}{print}\PY{p}{(}\PY{l+s+s2}{\PYZdq{}}\PY{l+s+s2}{Rozmiar danych}\PY{l+s+si}{\PYZob{}\PYZcb{}}\PY{l+s+s2}{: }\PY{l+s+s2}{\PYZdq{}}\PY{o}{.}\PY{n}{format}\PY{p}{(}\PY{n}{extra}\PY{p}{)}\PY{p}{,} \PY{n}{dataframe}\PY{o}{.}\PY{n}{shape}\PY{p}{)}     
        
        \PY{c+c1}{\PYZsh{} Ustalenie ścieżki do datasetu}
        \PY{n}{filename\PYZus{}forestfires} \PY{o}{=} \PY{l+s+s1}{\PYZsq{}}\PY{l+s+s1}{./forestfires.csv}\PY{l+s+s1}{\PYZsq{}}
        
        \PY{c+c1}{\PYZsh{} Wczytanie datasetu jako dataframe}
        \PY{n}{forestfires\PYZus{}dataframe} \PY{o}{=} \PY{n}{pd}\PY{o}{.}\PY{n}{read\PYZus{}csv}\PY{p}{(}\PY{n}{filename\PYZus{}forestfires}\PY{p}{,} \PY{n}{sep}\PY{o}{=}\PY{l+s+s2}{\PYZdq{}}\PY{l+s+s2}{;}\PY{l+s+s2}{\PYZdq{}}\PY{p}{)}
        
        \PY{c+c1}{\PYZsh{} Wyświetlenie dataframe}
        \PY{n}{display}\PY{p}{(}\PY{n}{forestfires\PYZus{}dataframe}\PY{p}{)}
\end{Verbatim}


    
    \begin{verbatim}
     X  Y month  day  FFMC    DMC     DC   ISI  temp  RH  wind  rain   area
0    7  5   mar  fri  86.2   26.2   94.3   5.1   8.2  51   6.7   0.0   0.00
1    7  4   oct  tue  90.6   35.4  669.1   6.7  18.0  33   0.9   0.0   0.00
2    7  4   oct  sat  90.6   43.7  686.9   6.7  14.6  33   1.3   0.0   0.00
3    8  6   mar  fri  91.7   33.3   77.5   9.0   8.3  97   4.0   0.2   0.00
4    8  6   mar  sun  89.3   51.3  102.2   9.6  11.4  99   1.8   0.0   0.00
5    8  6   aug  sun  92.3   85.3  488.0  14.7  22.2  29   5.4   0.0   0.00
6    8  6   aug  mon  92.3   88.9  495.6   8.5  24.1  27   3.1   0.0   0.00
7    8  6   aug  mon  91.5  145.4  608.2  10.7   8.0  86   2.2   0.0   0.00
8    8  6   sep  tue  91.0  129.5  692.6   7.0  13.1  63   5.4   0.0   0.00
9    7  5   sep  sat  92.5   88.0  698.6   7.1  22.8  40   4.0   0.0   0.00
10   7  5   sep  sat  92.5   88.0  698.6   7.1  17.8  51   7.2   0.0   0.00
11   7  5   sep  sat  92.8   73.2  713.0  22.6  19.3  38   4.0   0.0   0.00
12   6  5   aug  fri  63.5   70.8  665.3   0.8  17.0  72   6.7   0.0   0.00
13   6  5   sep  mon  90.9  126.5  686.5   7.0  21.3  42   2.2   0.0   0.00
14   6  5   sep  wed  92.9  133.3  699.6   9.2  26.4  21   4.5   0.0   0.00
15   6  5   sep  fri  93.3  141.2  713.9  13.9  22.9  44   5.4   0.0   0.00
16   5  5   mar  sat  91.7   35.8   80.8   7.8  15.1  27   5.4   0.0   0.00
17   8  5   oct  mon  84.9   32.8  664.2   3.0  16.7  47   4.9   0.0   0.00
18   6  4   mar  wed  89.2   27.9   70.8   6.3  15.9  35   4.0   0.0   0.00
19   6  4   apr  sat  86.3   27.4   97.1   5.1   9.3  44   4.5   0.0   0.00
20   6  4   sep  tue  91.0  129.5  692.6   7.0  18.3  40   2.7   0.0   0.00
21   5  4   sep  mon  91.8   78.5  724.3   9.2  19.1  38   2.7   0.0   0.00
22   7  4   jun  sun  94.3   96.3  200.0  56.1  21.0  44   4.5   0.0   0.00
23   7  4   aug  sat  90.2  110.9  537.4   6.2  19.5  43   5.8   0.0   0.00
24   7  4   aug  sat  93.5  139.4  594.2  20.3  23.7  32   5.8   0.0   0.00
25   7  4   aug  sun  91.4  142.4  601.4  10.6  16.3  60   5.4   0.0   0.00
26   7  4   sep  fri  92.4  117.9  668.0  12.2  19.0  34   5.8   0.0   0.00
27   7  4   sep  mon  90.9  126.5  686.5   7.0  19.4  48   1.3   0.0   0.00
28   6  3   sep  sat  93.4  145.4  721.4   8.1  30.2  24   2.7   0.0   0.00
29   6  3   sep  sun  93.5  149.3  728.6   8.1  22.8  39   3.6   0.0   0.00
..  .. ..   ...  ...   ...    ...    ...   ...   ...  ..   ...   ...    ...
487  5  4   aug  tue  95.1  141.3  605.8  17.7  26.4  34   3.6   0.0  16.40
488  4  4   aug  tue  95.1  141.3  605.8  17.7  19.4  71   7.6   0.0  46.70
489  4  4   aug  wed  95.1  141.3  605.8  17.7  20.6  58   1.3   0.0   0.00
490  4  4   aug  wed  95.1  141.3  605.8  17.7  28.7  33   4.0   0.0   0.00
491  4  4   aug  thu  95.8  152.0  624.1  13.8  32.4  21   4.5   0.0   0.00
492  1  3   aug  fri  95.9  158.0  633.6  11.3  32.4  27   2.2   0.0   0.00
493  1  3   aug  fri  95.9  158.0  633.6  11.3  27.5  29   4.5   0.0  43.32
494  6  6   aug  sat  96.0  164.0  643.0  14.0  30.8  30   4.9   0.0   8.59
495  6  6   aug  mon  96.2  175.5  661.8  16.8  23.9  42   2.2   0.0   0.00
496  4  5   aug  mon  96.2  175.5  661.8  16.8  32.6  26   3.1   0.0   2.77
497  3  4   aug  tue  96.1  181.1  671.2  14.3  32.3  27   2.2   0.0  14.68
498  6  5   aug  tue  96.1  181.1  671.2  14.3  33.3  26   2.7   0.0  40.54
499  7  5   aug  tue  96.1  181.1  671.2  14.3  27.3  63   4.9   6.4  10.82
500  8  6   aug  tue  96.1  181.1  671.2  14.3  21.6  65   4.9   0.8   0.00
501  7  5   aug  tue  96.1  181.1  671.2  14.3  21.6  65   4.9   0.8   0.00
502  4  4   aug  tue  96.1  181.1  671.2  14.3  20.7  69   4.9   0.4   0.00
503  2  4   aug  wed  94.5  139.4  689.1  20.0  29.2  30   4.9   0.0   1.95
504  4  3   aug  wed  94.5  139.4  689.1  20.0  28.9  29   4.9   0.0  49.59
505  1  2   aug  thu  91.0  163.2  744.4  10.1  26.7  35   1.8   0.0   5.80
506  1  2   aug  fri  91.0  166.9  752.6   7.1  18.5  73   8.5   0.0   0.00
507  2  4   aug  fri  91.0  166.9  752.6   7.1  25.9  41   3.6   0.0   0.00
508  1  2   aug  fri  91.0  166.9  752.6   7.1  25.9  41   3.6   0.0   0.00
509  5  4   aug  fri  91.0  166.9  752.6   7.1  21.1  71   7.6   1.4   2.17
510  6  5   aug  fri  91.0  166.9  752.6   7.1  18.2  62   5.4   0.0   0.43
511  8  6   aug  sun  81.6   56.7  665.6   1.9  27.8  35   2.7   0.0   0.00
512  4  3   aug  sun  81.6   56.7  665.6   1.9  27.8  32   2.7   0.0   6.44
513  2  4   aug  sun  81.6   56.7  665.6   1.9  21.9  71   5.8   0.0  54.29
514  7  4   aug  sun  81.6   56.7  665.6   1.9  21.2  70   6.7   0.0  11.16
515  1  4   aug  sat  94.4  146.0  614.7  11.3  25.6  42   4.0   0.0   0.00
516  6  3   nov  tue  79.5    3.0  106.7   1.1  11.8  31   4.5   0.0   0.00

[517 rows x 13 columns]
    \end{verbatim}

    
    \begin{Verbatim}[commandchars=\\\{\}]
{\color{incolor}In [{\color{incolor}3}]:} \PY{n}{dataframe\PYZus{}size\PYZus{}formated}\PY{p}{(}\PY{n}{forestfires\PYZus{}dataframe}\PY{p}{)}
\end{Verbatim}


    \begin{Verbatim}[commandchars=\\\{\}]
Rozmiar danych:  (517, 13)

    \end{Verbatim}

    Zbiór danych ma 517 wierszy i 13 kolumn (ostatnia kolumna to atrybut
decyzyjny, a pozostałe 12 kolumn to atrybuty warunkowe). W celu dalszego
zbadania datasetu i weryfikacji typów danych kategorycznych w każdej
kolumnie, wypisano unikalne wartości każdej kolumny. Sprawdzono również,
czy zbiór danych zawiera brakujące wartości lub niepotrzebne kolumny.

    \begin{Verbatim}[commandchars=\\\{\}]
{\color{incolor}In [{\color{incolor}4}]:} \PY{k}{def} \PY{n+nf}{attributes\PYZus{}count}\PY{p}{(}\PY{n}{dataframe}\PY{p}{)}\PY{p}{:}
            \PY{n+nb}{print}\PY{p}{(}\PY{l+s+s2}{\PYZdq{}}\PY{l+s+s2}{Liczba różnych wartości atrybutów dla każdej kolumny:}\PY{l+s+s2}{\PYZdq{}}\PY{p}{)}
            \PY{k}{for} \PY{n}{x} \PY{o+ow}{in} \PY{n}{dataframe}\PY{o}{.}\PY{n}{columns}\PY{p}{:}
                \PY{n}{uniq} \PY{o}{=} \PY{n}{dataframe}\PY{p}{[}\PY{n}{x}\PY{p}{]}\PY{o}{.}\PY{n}{unique}\PY{p}{(}\PY{p}{)}
                \PY{n+nb}{print}\PY{p}{(}\PY{l+s+s2}{\PYZdq{}}\PY{l+s+si}{\PYZob{}:\PYZgt{}8\PYZcb{}}\PY{l+s+s2}{: }\PY{l+s+si}{\PYZob{}:\PYZgt{}2\PYZcb{}}\PY{l+s+s2}{\PYZdq{}}\PY{o}{.}\PY{n}{format}\PY{p}{(}\PY{n}{x}\PY{p}{,} \PY{n}{uniq}\PY{o}{.}\PY{n}{shape}\PY{p}{[}\PY{l+m+mi}{0}\PY{p}{]}\PY{p}{)}\PY{p}{)}
                
        \PY{n}{attributes\PYZus{}count}\PY{p}{(}\PY{n}{forestfires\PYZus{}dataframe}\PY{p}{)}
\end{Verbatim}


    \begin{Verbatim}[commandchars=\\\{\}]
Liczba różnych wartości atrybutów dla każdej kolumny:
       X:  9
       Y:  7
   month: 12
     day:  7
    FFMC: 106
     DMC: 215
      DC: 219
     ISI: 119
    temp: 192
      RH: 75
    wind: 21
    rain:  7
    area: 251

    \end{Verbatim}

    Zauważono, że spośród 12 atrybutów warunkowych, 4 z nich mają liczbę
klas mniejeszą niż 10. Równiż atrybut month nie charakteryzuje się dużą
liczbą klas. Z uwagi na chęć wyeliminowania zalezności modelu od
położenia, dnia tygodnia i opadów deszczu, podjęto decyzję o usunięciu
kolumn o liczbie klas mniejszej równej 10, w tym celu utworzono poniższą
funkcję.
\vskip 0.2in
Wartości atrybutu month przekształcono w następujący sposób: `jan'=1,
`feb'=2, \ldots{}, `dec'=12.
    { \hspace*{\fill} \\}
    
    \begin{Verbatim}[commandchars=\\\{\}]
{\color{incolor}In [{\color{incolor}5}]:} \PY{k}{def} \PY{n+nf}{del\PYZus{}with\PYZus{}classes\PYZus{}no\PYZus{}less\PYZus{}than}\PY{p}{(}\PY{n}{dataframe}\PY{p}{,} \PY{n}{less}\PY{p}{)}\PY{p}{:}
            \PY{k}{for} \PY{n}{col} \PY{o+ow}{in} \PY{n}{dataframe}\PY{o}{.}\PY{n}{columns}\PY{o}{.}\PY{n}{values}\PY{p}{:}
                \PY{n}{col\PYZus{}unique} \PY{o}{=} \PY{n}{dataframe}\PY{p}{[}\PY{n}{col}\PY{p}{]}\PY{o}{.}\PY{n}{unique}\PY{p}{(}\PY{p}{)}
                \PY{k}{if} \PY{n+nb}{len}\PY{p}{(}\PY{n}{col\PYZus{}unique}\PY{p}{)} \PY{o}{\PYZlt{}}\PY{o}{=} \PY{n}{less}\PY{p}{:}
                    \PY{n+nb}{print}\PY{p}{(}\PY{l+s+s2}{\PYZdq{}}\PY{l+s+s2}{Usunięto kolumnę }\PY{l+s+s2}{\PYZsq{}}\PY{l+s+si}{\PYZob{}\PYZcb{}}\PY{l+s+s2}{\PYZsq{}}\PY{l+s+s2}{,która zawiera liczbę klas mnieszą równą }\PY{l+s+si}{\PYZob{}\PYZcb{}}\PY{l+s+s2}{: }\PY{l+s+si}{\PYZob{}\PYZcb{}}\PY{l+s+s2}{\PYZdq{}}\PY{o}{.}\PY{n}{format}\PY{p}{(}\PY{n}{col}\PY{p}{,} \PY{n}{less}\PY{p}{,} \PY{n}{col\PYZus{}unique}\PY{p}{)}\PY{p}{)}
                    \PY{n}{dataframe} \PY{o}{=} \PY{n}{dataframe}\PY{o}{.}\PY{n}{drop}\PY{p}{(}\PY{n}{col}\PY{p}{,} \PY{l+m+mi}{1}\PY{p}{)}
            \PY{k}{return} \PY{n}{dataframe}
\end{Verbatim}


    \begin{Verbatim}[commandchars=\\\{\}]
{\color{incolor}In [{\color{incolor}6}]:} \PY{n}{dataframe\PYZus{}size\PYZus{}formated}\PY{p}{(}\PY{n}{forestfires\PYZus{}dataframe}\PY{p}{,} \PY{l+s+s2}{\PYZdq{}}\PY{l+s+s2}{ przed usunięciem atrybutów}\PY{l+s+s2}{\PYZdq{}}\PY{p}{)}
        \PY{n}{forestfires\PYZus{}dataframe} \PY{o}{=} \PY{n}{del\PYZus{}with\PYZus{}classes\PYZus{}no\PYZus{}less\PYZus{}than}\PY{p}{(}\PY{n}{forestfires\PYZus{}dataframe}\PY{p}{,} \PY{l+m+mi}{10}\PY{p}{)}
        \PY{n}{dataframe\PYZus{}size\PYZus{}formated}\PY{p}{(}\PY{n}{forestfires\PYZus{}dataframe}\PY{p}{,} \PY{l+s+s2}{\PYZdq{}}\PY{l+s+s2}{ po usunięciu atrybutów}\PY{l+s+s2}{\PYZdq{}}\PY{p}{)}
        
        \PY{n}{forestfires\PYZus{}dataframe}\PY{o}{.}\PY{n}{month} \PY{o}{=} \PY{n}{forestfires\PYZus{}dataframe}\PY{o}{.}\PY{n}{month}\PY{o}{.}\PY{n}{map}\PY{p}{(}\PY{p}{\PYZob{}}
            \PY{l+s+s1}{\PYZsq{}}\PY{l+s+s1}{jan}\PY{l+s+s1}{\PYZsq{}}\PY{p}{:} \PY{l+m+mi}{1}\PY{p}{,}
            \PY{l+s+s1}{\PYZsq{}}\PY{l+s+s1}{feb}\PY{l+s+s1}{\PYZsq{}}\PY{p}{:} \PY{l+m+mi}{2}\PY{p}{,}
            \PY{l+s+s1}{\PYZsq{}}\PY{l+s+s1}{mar}\PY{l+s+s1}{\PYZsq{}}\PY{p}{:} \PY{l+m+mi}{3}\PY{p}{,}
            \PY{l+s+s1}{\PYZsq{}}\PY{l+s+s1}{apr}\PY{l+s+s1}{\PYZsq{}}\PY{p}{:} \PY{l+m+mi}{4}\PY{p}{,}
            \PY{l+s+s1}{\PYZsq{}}\PY{l+s+s1}{may}\PY{l+s+s1}{\PYZsq{}}\PY{p}{:} \PY{l+m+mi}{5}\PY{p}{,}
            \PY{l+s+s1}{\PYZsq{}}\PY{l+s+s1}{jun}\PY{l+s+s1}{\PYZsq{}}\PY{p}{:} \PY{l+m+mi}{6}\PY{p}{,}
            \PY{l+s+s1}{\PYZsq{}}\PY{l+s+s1}{jul}\PY{l+s+s1}{\PYZsq{}}\PY{p}{:} \PY{l+m+mi}{7}\PY{p}{,}
            \PY{l+s+s1}{\PYZsq{}}\PY{l+s+s1}{aug}\PY{l+s+s1}{\PYZsq{}}\PY{p}{:} \PY{l+m+mi}{8}\PY{p}{,}
            \PY{l+s+s1}{\PYZsq{}}\PY{l+s+s1}{sep}\PY{l+s+s1}{\PYZsq{}}\PY{p}{:} \PY{l+m+mi}{9}\PY{p}{,}
            \PY{l+s+s1}{\PYZsq{}}\PY{l+s+s1}{oct}\PY{l+s+s1}{\PYZsq{}}\PY{p}{:} \PY{l+m+mi}{10}\PY{p}{,}
            \PY{l+s+s1}{\PYZsq{}}\PY{l+s+s1}{nov}\PY{l+s+s1}{\PYZsq{}}\PY{p}{:} \PY{l+m+mi}{11}\PY{p}{,}
            \PY{l+s+s1}{\PYZsq{}}\PY{l+s+s1}{dec}\PY{l+s+s1}{\PYZsq{}}\PY{p}{:} \PY{l+m+mi}{12}\PY{p}{,}
        \PY{p}{\PYZcb{}}\PY{p}{)}
\end{Verbatim}


    \begin{Verbatim}[commandchars=\\\{\}]
Rozmiar danych przed usunięciem atrybutów:  (517, 13)
Usunięto kolumnę 'X',która zawiera liczbę klas mnieszą równą 10: [7 8 6 5 4 2 9 1 3]
Usunięto kolumnę 'Y',która zawiera liczbę klas mnieszą równą 10: [5 4 6 3 2 9 8]
Usunięto kolumnę 'day',która zawiera liczbę klas mnieszą równą 10: ['fri' 'tue' 'sat' 'sun' 'mon' 'wed' 'thu']
Usunięto kolumnę 'rain',która zawiera liczbę klas mnieszą równą 10: [ 0.   0.2  1.   6.4  0.8  0.4  1.4]
Rozmiar danych po usunięciu atrybutów:  (517, 9)

    \end{Verbatim}
    { \hspace*{\fill} \\}
    
    \begin{Verbatim}[commandchars=\\\{\}]
{\color{incolor}In [{\color{incolor}7}]:} \PY{n}{attributes\PYZus{}count}\PY{p}{(}\PY{n}{forestfires\PYZus{}dataframe}\PY{p}{)}
\end{Verbatim}


    \begin{Verbatim}[commandchars=\\\{\}]
Liczba różnych wartości atrybutów dla każdej kolumny:
   month: 12
    FFMC: 106
     DMC: 215
      DC: 219
     ISI: 119
    temp: 192
      RH: 75
    wind: 21
    area: 251

    \end{Verbatim}

    Z uwagi na brak danych kategorycznych w oczyszczonym zbiorze, kodowanie
wartości atrybutów (kolumn) nie jest konieczne. Dokonać podziału danych
na atrybuty warunkowe (zmienna X) i decyzyjne (zmienna Y).

    \begin{Verbatim}[commandchars=\\\{\}]
{\color{incolor}In [{\color{incolor}8}]:} \PY{n}{X} \PY{o}{=} \PY{n}{forestfires\PYZus{}dataframe}\PY{o}{.}\PY{n}{drop}\PY{p}{(}\PY{p}{[}\PY{l+s+s1}{\PYZsq{}}\PY{l+s+s1}{area}\PY{l+s+s1}{\PYZsq{}}\PY{p}{]}\PY{p}{,} \PY{n}{axis}\PY{o}{=}\PY{l+m+mi}{1}\PY{p}{)}
        \PY{n}{Y} \PY{o}{=} \PY{n}{forestfires\PYZus{}dataframe}\PY{p}{[}\PY{l+s+s1}{\PYZsq{}}\PY{l+s+s1}{area}\PY{l+s+s1}{\PYZsq{}}\PY{p}{]}
\end{Verbatim}


    Kolejnym podziałem, który należy wykonać, jest podział danych na część
treningową i testową. Założono, że rozmiar części testowej będzie
wynosił 33\% wszystkich danych. W celu zachowania powtarzalności wyników
parametr random\_state ustawiono na wartość 34 (ustawienie innej
wartości bedzie powodowało wygenerowanie innego podziału danych i innego
drzewa decyzyjnego).

    \begin{Verbatim}[commandchars=\\\{\}]
{\color{incolor}In [{\color{incolor}9}]:} \PY{n}{random\PYZus{}state} \PY{o}{=} \PY{l+m+mi}{34}
        
        \PY{n}{X\PYZus{}train}\PY{p}{,} \PY{n}{X\PYZus{}test} \PY{p}{,}\PY{n}{Y\PYZus{}train}\PY{p}{,} \PY{n}{Y\PYZus{}test} \PY{o}{=} \PY{n}{train\PYZus{}test\PYZus{}split}\PY{p}{(}\PY{n}{X}\PY{p}{,} \PY{n}{Y}\PY{p}{,} \PY{n}{test\PYZus{}size} \PY{o}{=} \PY{l+m+mf}{0.33}\PY{p}{,} \PY{n}{random\PYZus{}state}\PY{o}{=}\PY{n}{random\PYZus{}state}\PY{p}{)}
\end{Verbatim}


    W oparciu o dane zbudowano drzewo decyzyjne. Do tego celu utworzno
specjalną funkcję.

    \begin{Verbatim}[commandchars=\\\{\}]
{\color{incolor}In [{\color{incolor}10}]:} \PY{k}{def} \PY{n+nf}{build\PYZus{}tree}\PY{p}{(}\PY{n}{X}\PY{p}{,} \PY{n}{X\PYZus{}train}\PY{p}{,} \PY{n}{X\PYZus{}test}\PY{p}{,} \PY{n}{Y\PYZus{}train}\PY{p}{,} \PY{n}{Y\PYZus{}test}\PY{p}{,} \PY{n}{random\PYZus{}state}\PY{p}{,} \PY{o}{*}\PY{o}{*}\PY{n}{kwargs}\PY{p}{)}\PY{p}{:}
             \PY{n}{regr} \PY{o}{=} \PY{n}{tree}\PY{o}{.}\PY{n}{DecisionTreeRegressor}\PY{p}{(}\PY{n}{random\PYZus{}state}\PY{o}{=}\PY{n}{random\PYZus{}state}\PY{p}{,} \PY{o}{*}\PY{o}{*}\PY{n}{kwargs}\PY{p}{)}
             \PY{n}{regr} \PY{o}{=} \PY{n}{regr}\PY{o}{.}\PY{n}{fit}\PY{p}{(}\PY{n}{X\PYZus{}train}\PY{p}{,} \PY{n}{Y\PYZus{}train}\PY{p}{)}
         
             \PY{n}{dot\PYZus{}data} \PY{o}{=} \PY{n}{tree}\PY{o}{.}\PY{n}{export\PYZus{}graphviz}\PY{p}{(}\PY{n}{regr}\PY{p}{,} \PY{n}{out\PYZus{}file}\PY{o}{=}\PY{k+kc}{None}\PY{p}{,}  
                                             \PY{n}{feature\PYZus{}names}\PY{o}{=}\PY{n}{X}\PY{o}{.}\PY{n}{columns}\PY{p}{,} 
                                             \PY{n}{filled}\PY{o}{=}\PY{k+kc}{True}\PY{p}{,} \PY{n}{rounded}\PY{o}{=}\PY{k+kc}{True}\PY{p}{,}  
                                             \PY{n}{special\PYZus{}characters}\PY{o}{=}\PY{k+kc}{True}\PY{p}{)}  
             \PY{n}{graph} \PY{o}{=} \PY{n}{pydotplus}\PY{o}{.}\PY{n}{graph\PYZus{}from\PYZus{}dot\PYZus{}data}\PY{p}{(}\PY{n}{dot\PYZus{}data}\PY{p}{)}  
             \PY{n}{display}\PY{p}{(}\PY{n}{Image}\PY{p}{(}\PY{n}{graph}\PY{o}{.}\PY{n}{create\PYZus{}png}\PY{p}{(}\PY{p}{)}\PY{p}{)}\PY{p}{)}
                 
             \PY{k}{return} \PY{n}{regr}
\end{Verbatim}


    \begin{Verbatim}[commandchars=\\\{\}]
{\color{incolor}In [{\color{incolor}11}]:} \PY{n}{regr} \PY{o}{=} \PY{n}{build\PYZus{}tree}\PY{p}{(}\PY{n}{X}\PY{p}{,} \PY{n}{X\PYZus{}train}\PY{p}{,} \PY{n}{X\PYZus{}test}\PY{p}{,} \PY{n}{Y\PYZus{}train}\PY{p}{,} \PY{n}{Y\PYZus{}test}\PY{p}{,} \PY{n}{random\PYZus{}state}\PY{p}{,} \PY{n}{max\PYZus{}depth}\PY{o}{=}\PY{k+kc}{None}\PY{p}{,} \PY{n}{criterion}\PY{o}{=}\PY{l+s+s1}{\PYZsq{}}\PY{l+s+s1}{mae}\PY{l+s+s1}{\PYZsq{}}\PY{p}{)}
\end{Verbatim}
    { \hspace*{\fill} \\}
    

    \begin{center}
    \adjustimage{max size={1\linewidth}{1\paperheight}}{output_17_0.png}
    \end{center}
    { \hspace*{\fill} \\}
        { \hspace*{\fill} \\}
        { \hspace*{\fill} \\}
    
    Drzewo wygenerowane bez przycinania oraz innych optymalizacji jest zbyt
duże.

    \begin{Verbatim}[commandchars=\\\{\}]
{\color{incolor}In [{\color{incolor}12}]:} \PY{k}{def} \PY{n+nf}{check\PYZus{}tree\PYZus{}mae}\PY{p}{(}\PY{n}{regr}\PY{p}{,} \PY{n}{X\PYZus{}train}\PY{p}{,} \PY{n}{X\PYZus{}test}\PY{p}{,} \PY{n}{Y\PYZus{}train}\PY{p}{,} \PY{n}{Y\PYZus{}test}\PY{p}{,} \PY{n}{print\PYZus{}score}\PY{o}{=}\PY{k+kc}{True}\PY{p}{)}\PY{p}{:}
             \PY{n}{preds}\PY{o}{=}\PY{n}{regr}\PY{o}{.}\PY{n}{predict}\PY{p}{(}\PY{n}{X\PYZus{}test}\PY{p}{)}
             
             \PY{n}{line\PYZus{}pred}\PY{p}{,} \PY{o}{=} \PY{n}{plt}\PY{o}{.}\PY{n}{plot}\PY{p}{(}\PY{n}{preds}\PY{p}{,} \PY{n}{label}\PY{o}{=}\PY{l+s+s1}{\PYZsq{}}\PY{l+s+s1}{Wartość przewidziana Y\PYZus{}test}\PY{l+s+s1}{\PYZsq{}}\PY{p}{)}
             \PY{n}{line\PYZus{}real}\PY{p}{,} \PY{o}{=} \PY{n}{plt}\PY{o}{.}\PY{n}{plot}\PY{p}{(}\PY{n}{Y\PYZus{}test}\PY{o}{.}\PY{n}{values}\PY{p}{,} \PY{n}{label}\PY{o}{=}\PY{l+s+s1}{\PYZsq{}}\PY{l+s+s1}{Wartość rzeczywista Y\PYZus{}test}\PY{l+s+s1}{\PYZsq{}}\PY{p}{)}
             \PY{n}{plt}\PY{o}{.}\PY{n}{legend}\PY{p}{(}\PY{n}{handles}\PY{o}{=}\PY{p}{[}\PY{n}{line\PYZus{}pred}\PY{p}{,} \PY{n}{line\PYZus{}real}\PY{p}{]}\PY{p}{)}
           
             \PY{n}{mae} \PY{o}{=} \PY{n}{mean\PYZus{}absolute\PYZus{}error}\PY{p}{(}\PY{n}{Y\PYZus{}test}\PY{p}{,} \PY{n}{preds}\PY{p}{)}
             \PY{k}{if} \PY{n}{print\PYZus{}score}\PY{p}{:}
                 \PY{n+nb}{print}\PY{p}{(}\PY{l+s+s1}{\PYZsq{}}\PY{l+s+s1}{Mean absolute error: }\PY{l+s+si}{\PYZpc{}.3f}\PY{l+s+s1}{\PYZsq{}} \PY{o}{\PYZpc{}} \PY{n}{mae}\PY{p}{)}
             \PY{k}{return} \PY{n}{mae}
         
         \PY{k}{def} \PY{n+nf}{check\PYZus{}tree\PYZus{}cross\PYZus{}val\PYZus{}score}\PY{p}{(}\PY{n}{regr}\PY{p}{,} \PY{n}{X}\PY{p}{,} \PY{n}{Y}\PY{p}{,} \PY{n}{print\PYZus{}score}\PY{o}{=}\PY{k+kc}{True}\PY{p}{)}\PY{p}{:}
             \PY{n}{scores} \PY{o}{=} \PY{n}{cross\PYZus{}val\PYZus{}score}\PY{p}{(}\PY{n}{regr}\PY{p}{,} \PY{n}{X}\PY{p}{,} \PY{n}{Y}\PY{p}{,} \PY{n}{cv}\PY{o}{=}\PY{l+m+mi}{3}\PY{p}{,} \PY{n}{scoring}\PY{o}{=}\PY{l+s+s1}{\PYZsq{}}\PY{l+s+s1}{neg\PYZus{}mean\PYZus{}squared\PYZus{}error}\PY{l+s+s1}{\PYZsq{}}\PY{p}{)}
             \PY{n}{current\PYZus{}score} \PY{o}{=} \PY{n}{np}\PY{o}{.}\PY{n}{mean}\PY{p}{(}\PY{n}{np}\PY{o}{.}\PY{n}{sqrt}\PY{p}{(}\PY{o}{\PYZhy{}}\PY{n}{scores}\PY{p}{)}\PY{p}{)}
         
             \PY{k}{if} \PY{n}{print\PYZus{}score}\PY{p}{:}
                 \PY{n+nb}{print}\PY{p}{(}\PY{l+s+s1}{\PYZsq{}}\PY{l+s+s1}{Dokładność pomiędzy Y\PYZus{}pred oraz Y\PYZus{}test (neg\PYZus{}mean\PYZus{}squared\PYZus{}error): }\PY{l+s+si}{\PYZpc{}.3f}\PY{l+s+s1}{\PYZsq{}} \PY{o}{\PYZpc{}} \PY{n}{current\PYZus{}score}\PY{p}{)}
             \PY{k}{return} \PY{n}{current\PYZus{}score}              
\end{Verbatim}


    Metody mean absolute error (MAE) oraz mean squared error (MSE) zostały
wykorzystane do określenia dokładności między Y\_pred a Y\_test. Im
mniejsza wartość obydwu parametrów tym drzewo decyzyjne jest lepiej
dopasowane do danych testowych. MAE jest miarą różnicy między dwiema
zmiennymi ciągłymi. Z kolei MSE jest wartością oczekiwaną kwadratu
„błędu'', czyli różnicy pomiędzy estymatorem i wartością estymowaną.

    \begin{Verbatim}[commandchars=\\\{\}]
{\color{incolor}In [{\color{incolor}13}]:} \PY{n}{check\PYZus{}tree\PYZus{}mae}\PY{p}{(}\PY{n}{regr}\PY{p}{,} \PY{n}{X\PYZus{}train}\PY{p}{,} \PY{n}{X\PYZus{}test}\PY{p}{,} \PY{n}{Y\PYZus{}train}\PY{p}{,} \PY{n}{Y\PYZus{}test}\PY{p}{)}
         \PY{n}{check\PYZus{}tree\PYZus{}cross\PYZus{}val\PYZus{}score}\PY{p}{(}\PY{n}{regr}\PY{p}{,} \PY{n}{X}\PY{p}{,} \PY{n}{Y}\PY{p}{)}
         
         \PY{n+nb}{print}\PY{p}{(}\PY{l+s+s2}{\PYZdq{}}\PY{l+s+s2}{\PYZdq{}}\PY{p}{)}
\end{Verbatim}


    \begin{Verbatim}[commandchars=\\\{\}]
Mean absolute error: 24.524
Dokładność pomiędzy Y\_pred oraz Y\_test (neg\_mean\_squared\_error): 108.680


    \end{Verbatim}

    \begin{center}
    \adjustimage{max size={0.9\linewidth}{0.9\paperheight}}{output_21_1.png}
    \end{center}
\    
    Z wartości Mean absolute error widać, że błąd drzewa z maksymalnie 10
poziomami wynosi średnio około 24,5 hektara lasu na przewidywanie.
Również bład MSE jest dość duży.

    \begin{Verbatim}[commandchars=\\\{\}]
{\color{incolor}In [{\color{incolor}14}]:} \PY{k}{def} \PY{n+nf}{attribute\PYZus{}importance}\PY{p}{(}\PY{n}{regr}\PY{p}{,} \PY{n}{X}\PY{p}{)}\PY{p}{:}
             \PY{n}{attrs} \PY{o}{=} \PY{n}{X}\PY{o}{.}\PY{n}{columns}\PY{o}{.}\PY{n}{values}
             \PY{n}{attr\PYZus{}importance} \PY{o}{=} \PY{n}{regr}\PY{o}{.}\PY{n}{feature\PYZus{}importances\PYZus{}}
             \PY{n}{sorted\PYZus{}attr\PYZus{}importance} \PY{o}{=} \PY{n}{np}\PY{o}{.}\PY{n}{argsort}\PY{p}{(}\PY{n}{attr\PYZus{}importance}\PY{p}{)}
             \PY{n}{range\PYZus{}sorted\PYZus{}attr\PYZus{}importance} \PY{o}{=} \PY{n+nb}{range}\PY{p}{(}\PY{n+nb}{len}\PY{p}{(}\PY{n}{sorted\PYZus{}attr\PYZus{}importance}\PY{p}{)}\PY{p}{)}
             
             \PY{n}{plt}\PY{o}{.}\PY{n}{figure}\PY{p}{(}\PY{n}{figsize}\PY{o}{=}\PY{p}{(}\PY{l+m+mi}{8}\PY{p}{,} \PY{l+m+mi}{7}\PY{p}{)}\PY{p}{)}
             \PY{n}{plt}\PY{o}{.}\PY{n}{barh}\PY{p}{(}\PY{n}{range\PYZus{}sorted\PYZus{}attr\PYZus{}importance}\PY{p}{,} \PY{n}{attr\PYZus{}importance}\PY{p}{[}\PY{n}{sorted\PYZus{}attr\PYZus{}importance}\PY{p}{]}\PY{p}{,} \PY{n}{color}\PY{o}{=}\PY{l+s+s1}{\PYZsq{}}\PY{l+s+s1}{\PYZsh{}33cc33}\PY{l+s+s1}{\PYZsq{}}\PY{p}{)}
             \PY{n}{plt}\PY{o}{.}\PY{n}{yticks}\PY{p}{(}\PY{n}{range\PYZus{}sorted\PYZus{}attr\PYZus{}importance}\PY{p}{,} \PY{n}{attrs}\PY{p}{[}\PY{n}{sorted\PYZus{}attr\PYZus{}importance}\PY{p}{]}\PY{p}{)}
             \PY{n}{plt}\PY{o}{.}\PY{n}{xlabel}\PY{p}{(}\PY{l+s+s1}{\PYZsq{}}\PY{l+s+s1}{Importance}\PY{l+s+s1}{\PYZsq{}}\PY{p}{)}
             \PY{n}{plt}\PY{o}{.}\PY{n}{title}\PY{p}{(}\PY{l+s+s1}{\PYZsq{}}\PY{l+s+s1}{Attribute importances}\PY{l+s+s1}{\PYZsq{}}\PY{p}{)}
             \PY{n}{plt}\PY{o}{.}\PY{n}{draw}\PY{p}{(}\PY{p}{)}
             \PY{n}{plt}\PY{o}{.}\PY{n}{show}\PY{p}{(}\PY{p}{)}
\end{Verbatim}


    \begin{Verbatim}[commandchars=\\\{\}]
{\color{incolor}In [{\color{incolor}15}]:} \PY{n}{attribute\PYZus{}importance}\PY{p}{(}\PY{n}{regr}\PY{p}{,} \PY{n}{X}\PY{p}{)}
\end{Verbatim}


    \begin{center}
    \adjustimage{max size={0.75\paperheight}}{output_24_0.png}
    \end{center}
    
    Zauważono, że najważnijszymi atrybutami są: month oraz temp. Mniejsze
znaczenie mają pozostałe atrybuty. Wygenerowane drzewo decyzyjne posiada
głębokość powyżej 15, jest to zdecydowanie zbyt dużo. Optymalne drzewo
przedstawiono poniżej.

    \begin{Verbatim}[commandchars=\\\{\}]
{\color{incolor}In [{\color{incolor}16}]:} \PY{n}{regr} \PY{o}{=} \PY{n}{build\PYZus{}tree}\PY{p}{(}\PY{n}{X}\PY{p}{,} \PY{n}{X\PYZus{}train}\PY{p}{,} \PY{n}{X\PYZus{}test}\PY{p}{,} \PY{n}{Y\PYZus{}train}\PY{p}{,} \PY{n}{Y\PYZus{}test}\PY{p}{,} \PY{n}{random\PYZus{}state}\PY{p}{,} \PY{n}{max\PYZus{}depth}\PY{o}{=}\PY{l+m+mi}{6}\PY{p}{,} \PY{n}{min\PYZus{}samples\PYZus{}leaf} \PY{o}{=} \PY{l+m+mi}{8}\PY{p}{,} \PY{n}{criterion}\PY{o}{=}\PY{l+s+s1}{\PYZsq{}}\PY{l+s+s1}{mae}\PY{l+s+s1}{\PYZsq{}}\PY{p}{)}
\end{Verbatim}


    \begin{center}
    \adjustimage{max size={0.9\linewidth}{0.9\paperheight}}{output_26_0.png}
    \end{center}
    { \hspace*{\fill} \\}
    
    \begin{Verbatim}[commandchars=\\\{\}]
{\color{incolor}In [{\color{incolor}17}]:} \PY{n}{check\PYZus{}tree\PYZus{}mae}\PY{p}{(}\PY{n}{regr}\PY{p}{,} \PY{n}{X\PYZus{}train}\PY{p}{,} \PY{n}{X\PYZus{}test}\PY{p}{,} \PY{n}{Y\PYZus{}train}\PY{p}{,} \PY{n}{Y\PYZus{}test}\PY{p}{)}
         \PY{n}{check\PYZus{}tree\PYZus{}cross\PYZus{}val\PYZus{}score}\PY{p}{(}\PY{n}{regr}\PY{p}{,} \PY{n}{X}\PY{p}{,} \PY{n}{Y}\PY{p}{)}
         
         \PY{n+nb}{print}\PY{p}{(}\PY{l+s+s2}{\PYZdq{}}\PY{l+s+s2}{\PYZdq{}}\PY{p}{)}
\end{Verbatim}


    \begin{Verbatim}[commandchars=\\\{\}]
Mean absolute error: 8.706
Dokładność pomiędzy Y\_pred oraz Y\_test (neg\_mean\_squared\_error): 58.401


    \end{Verbatim}

    \begin{center}
    \adjustimage{max size={0.9\linewidth}{0.9\paperheight}}{output_27_1.png}
    \end{center}

    Z wartości Mean absolute error widać, że błąd drzewa z maksymalnie 10
poziomami wynosi średnio około 8,7 hektara lasu na przewidywanie.
Stwierdzono, że drzewo jest dość dobrze dopasowane do danych w
porównaniu do poprzedniego drzewa.

    \begin{Verbatim}[commandchars=\\\{\}]
{\color{incolor}In [{\color{incolor}18}]:} \PY{n}{attribute\PYZus{}importance}\PY{p}{(}\PY{n}{regr}\PY{p}{,} \PY{n}{X}\PY{p}{)}
\end{Verbatim}


    \begin{center}
    \adjustimage{max size={0.7\paperheight}}{output_29_0.png}
    \end{center}

    
    Wygenerowane drzewo decyzyjne posiada głębokość wynoszącą 6 (drzewo bez
ograniczenia wysokości głębokość powyżej 15). Zauważono, że
najważnijszymi najważnijesze atrybuty nie zmieniły się w stosunku do
poprzedniego drzewa.


    % Add a bibliography block to the postdoc
    
    
    
    \end{document}
