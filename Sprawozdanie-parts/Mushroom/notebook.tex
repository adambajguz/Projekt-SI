
% Default to the notebook output style

    


% Inherit from the specified cell style.




    
\documentclass[10pt, a4paper, landscape]{article}
\setcounter{page}{12}
    
    
    \usepackage[T1]{fontenc}
    % Nicer default font (+ math font) than Computer Modern for most use cases
    \usepackage{mathpazo}

    % Basic figure setup, for now with no caption control since it's done
    % automatically by Pandoc (which extracts ![](path) syntax from Markdown).
    \usepackage{graphicx}
    % We will generate all images so they have a width \maxwidth. This means
    % that they will get their normal width if they fit onto the page, but
    % are scaled down if they would overflow the margins.
    \makeatletter
    \def\maxwidth{\ifdim\Gin@nat@width>\linewidth\linewidth
    \else\Gin@nat@width\fi}
    \makeatother
    \let\Oldincludegraphics\includegraphics
    % Set max figure width to be 80% of text width, for now hardcoded.
    \renewcommand{\includegraphics}[1]{\Oldincludegraphics[width=.8\maxwidth]{#1}}
    % Ensure that by default, figures have no caption (until we provide a
    % proper Figure object with a Caption API and a way to capture that
    % in the conversion process - todo).
    \usepackage{caption}
    \DeclareCaptionLabelFormat{nolabel}{}
    \captionsetup{labelformat=nolabel}

    \usepackage{adjustbox} % Used to constrain images to a maximum size 
    \usepackage{xcolor} % Allow colors to be defined
    \usepackage{enumerate} % Needed for markdown enumerations to work
    \usepackage{geometry} % Used to adjust the document margins
    \usepackage{amsmath} % Equations
    \usepackage{amssymb} % Equations
    \usepackage{textcomp} % defines textquotesingle
    % Hack from http://tex.stackexchange.com/a/47451/13684:
    \AtBeginDocument{%
        \def\PYZsq{\textquotesingle}% Upright quotes in Pygmentized code
    }
    \usepackage{upquote} % Upright quotes for verbatim code
    \usepackage{eurosym} % defines \euro
    \usepackage[mathletters]{ucs} % Extended unicode (utf-8) support
    \usepackage[utf8x]{inputenc} % Allow utf-8 characters in the tex document
    \usepackage{fancyvrb} % verbatim replacement that allows latex
    \usepackage{grffile} % extends the file name processing of package graphics 
                         % to support a larger range 
    % The hyperref package gives us a pdf with properly built
    % internal navigation ('pdf bookmarks' for the table of contents,
    % internal cross-reference links, web links for URLs, etc.)
    \usepackage{hyperref}
    \usepackage{longtable} % longtable support required by pandoc >1.10
    \usepackage{booktabs}  % table support for pandoc > 1.12.2
    \usepackage[inline]{enumitem} % IRkernel/repr support (it uses the enumerate* environment)
    \usepackage[normalem]{ulem} % ulem is needed to support strikethroughs (\sout)
                                % normalem makes italics be italics, not underlines
    

    
    
    % Colors for the hyperref package
    \definecolor{urlcolor}{rgb}{0,.145,.698}
    \definecolor{linkcolor}{rgb}{.71,0.21,0.01}
    \definecolor{citecolor}{rgb}{.12,.54,.11}

    % ANSI colors
    \definecolor{ansi-black}{HTML}{3E424D}
    \definecolor{ansi-black-intense}{HTML}{282C36}
    \definecolor{ansi-red}{HTML}{E75C58}
    \definecolor{ansi-red-intense}{HTML}{B22B31}
    \definecolor{ansi-green}{HTML}{00A250}
    \definecolor{ansi-green-intense}{HTML}{007427}
    \definecolor{ansi-yellow}{HTML}{DDB62B}
    \definecolor{ansi-yellow-intense}{HTML}{B27D12}
    \definecolor{ansi-blue}{HTML}{208FFB}
    \definecolor{ansi-blue-intense}{HTML}{0065CA}
    \definecolor{ansi-magenta}{HTML}{D160C4}
    \definecolor{ansi-magenta-intense}{HTML}{A03196}
    \definecolor{ansi-cyan}{HTML}{60C6C8}
    \definecolor{ansi-cyan-intense}{HTML}{258F8F}
    \definecolor{ansi-white}{HTML}{C5C1B4}
    \definecolor{ansi-white-intense}{HTML}{A1A6B2}

    % commands and environments needed by pandoc snippets
    % extracted from the output of `pandoc -s`
    \providecommand{\tightlist}{%
      \setlength{\itemsep}{0pt}\setlength{\parskip}{0pt}}
    \DefineVerbatimEnvironment{Highlighting}{Verbatim}{commandchars=\\\{\}}
    % Add ',fontsize=\small' for more characters per line
    \newenvironment{Shaded}{}{}
    \newcommand{\KeywordTok}[1]{\textcolor[rgb]{0.00,0.44,0.13}{\textbf{{#1}}}}
    \newcommand{\DataTypeTok}[1]{\textcolor[rgb]{0.56,0.13,0.00}{{#1}}}
    \newcommand{\DecValTok}[1]{\textcolor[rgb]{0.25,0.63,0.44}{{#1}}}
    \newcommand{\BaseNTok}[1]{\textcolor[rgb]{0.25,0.63,0.44}{{#1}}}
    \newcommand{\FloatTok}[1]{\textcolor[rgb]{0.25,0.63,0.44}{{#1}}}
    \newcommand{\CharTok}[1]{\textcolor[rgb]{0.25,0.44,0.63}{{#1}}}
    \newcommand{\StringTok}[1]{\textcolor[rgb]{0.25,0.44,0.63}{{#1}}}
    \newcommand{\CommentTok}[1]{\textcolor[rgb]{0.38,0.63,0.69}{\textit{{#1}}}}
    \newcommand{\OtherTok}[1]{\textcolor[rgb]{0.00,0.44,0.13}{{#1}}}
    \newcommand{\AlertTok}[1]{\textcolor[rgb]{1.00,0.00,0.00}{\textbf{{#1}}}}
    \newcommand{\FunctionTok}[1]{\textcolor[rgb]{0.02,0.16,0.49}{{#1}}}
    \newcommand{\RegionMarkerTok}[1]{{#1}}
    \newcommand{\ErrorTok}[1]{\textcolor[rgb]{1.00,0.00,0.00}{\textbf{{#1}}}}
    \newcommand{\NormalTok}[1]{{#1}}
    
    % Additional commands for more recent versions of Pandoc
    \newcommand{\ConstantTok}[1]{\textcolor[rgb]{0.53,0.00,0.00}{{#1}}}
    \newcommand{\SpecialCharTok}[1]{\textcolor[rgb]{0.25,0.44,0.63}{{#1}}}
    \newcommand{\VerbatimStringTok}[1]{\textcolor[rgb]{0.25,0.44,0.63}{{#1}}}
    \newcommand{\SpecialStringTok}[1]{\textcolor[rgb]{0.73,0.40,0.53}{{#1}}}
    \newcommand{\ImportTok}[1]{{#1}}
    \newcommand{\DocumentationTok}[1]{\textcolor[rgb]{0.73,0.13,0.13}{\textit{{#1}}}}
    \newcommand{\AnnotationTok}[1]{\textcolor[rgb]{0.38,0.63,0.69}{\textbf{\textit{{#1}}}}}
    \newcommand{\CommentVarTok}[1]{\textcolor[rgb]{0.38,0.63,0.69}{\textbf{\textit{{#1}}}}}
    \newcommand{\VariableTok}[1]{\textcolor[rgb]{0.10,0.09,0.49}{{#1}}}
    \newcommand{\ControlFlowTok}[1]{\textcolor[rgb]{0.00,0.44,0.13}{\textbf{{#1}}}}
    \newcommand{\OperatorTok}[1]{\textcolor[rgb]{0.40,0.40,0.40}{{#1}}}
    \newcommand{\BuiltInTok}[1]{{#1}}
    \newcommand{\ExtensionTok}[1]{{#1}}
    \newcommand{\PreprocessorTok}[1]{\textcolor[rgb]{0.74,0.48,0.00}{{#1}}}
    \newcommand{\AttributeTok}[1]{\textcolor[rgb]{0.49,0.56,0.16}{{#1}}}
    \newcommand{\InformationTok}[1]{\textcolor[rgb]{0.38,0.63,0.69}{\textbf{\textit{{#1}}}}}
    \newcommand{\WarningTok}[1]{\textcolor[rgb]{0.38,0.63,0.69}{\textbf{\textit{{#1}}}}}
    
    
    % Define a nice break command that doesn't care if a line doesn't already
    % exist.
    \def\br{\hspace*{\fill} \\* }
    % Math Jax compatability definitions
    \def\gt{>}
    \def\lt{<}
    % Document parameters
    \title{MushroomDT}
    
    
    

    % Pygments definitions
    
\makeatletter
\def\PY@reset{\let\PY@it=\relax \let\PY@bf=\relax%
    \let\PY@ul=\relax \let\PY@tc=\relax%
    \let\PY@bc=\relax \let\PY@ff=\relax}
\def\PY@tok#1{\csname PY@tok@#1\endcsname}
\def\PY@toks#1+{\ifx\relax#1\empty\else%
    \PY@tok{#1}\expandafter\PY@toks\fi}
\def\PY@do#1{\PY@bc{\PY@tc{\PY@ul{%
    \PY@it{\PY@bf{\PY@ff{#1}}}}}}}
\def\PY#1#2{\PY@reset\PY@toks#1+\relax+\PY@do{#2}}

\expandafter\def\csname PY@tok@gr\endcsname{\def\PY@tc##1{\textcolor[rgb]{1.00,0.00,0.00}{##1}}}
\expandafter\def\csname PY@tok@gp\endcsname{\let\PY@bf=\textbf\def\PY@tc##1{\textcolor[rgb]{0.00,0.00,0.50}{##1}}}
\expandafter\def\csname PY@tok@sa\endcsname{\def\PY@tc##1{\textcolor[rgb]{0.73,0.13,0.13}{##1}}}
\expandafter\def\csname PY@tok@cpf\endcsname{\let\PY@it=\textit\def\PY@tc##1{\textcolor[rgb]{0.25,0.50,0.50}{##1}}}
\expandafter\def\csname PY@tok@nb\endcsname{\def\PY@tc##1{\textcolor[rgb]{0.00,0.50,0.00}{##1}}}
\expandafter\def\csname PY@tok@ni\endcsname{\let\PY@bf=\textbf\def\PY@tc##1{\textcolor[rgb]{0.60,0.60,0.60}{##1}}}
\expandafter\def\csname PY@tok@mo\endcsname{\def\PY@tc##1{\textcolor[rgb]{0.40,0.40,0.40}{##1}}}
\expandafter\def\csname PY@tok@s2\endcsname{\def\PY@tc##1{\textcolor[rgb]{0.73,0.13,0.13}{##1}}}
\expandafter\def\csname PY@tok@sx\endcsname{\def\PY@tc##1{\textcolor[rgb]{0.00,0.50,0.00}{##1}}}
\expandafter\def\csname PY@tok@mh\endcsname{\def\PY@tc##1{\textcolor[rgb]{0.40,0.40,0.40}{##1}}}
\expandafter\def\csname PY@tok@kn\endcsname{\let\PY@bf=\textbf\def\PY@tc##1{\textcolor[rgb]{0.00,0.50,0.00}{##1}}}
\expandafter\def\csname PY@tok@se\endcsname{\let\PY@bf=\textbf\def\PY@tc##1{\textcolor[rgb]{0.73,0.40,0.13}{##1}}}
\expandafter\def\csname PY@tok@ss\endcsname{\def\PY@tc##1{\textcolor[rgb]{0.10,0.09,0.49}{##1}}}
\expandafter\def\csname PY@tok@gs\endcsname{\let\PY@bf=\textbf}
\expandafter\def\csname PY@tok@gi\endcsname{\def\PY@tc##1{\textcolor[rgb]{0.00,0.63,0.00}{##1}}}
\expandafter\def\csname PY@tok@kc\endcsname{\let\PY@bf=\textbf\def\PY@tc##1{\textcolor[rgb]{0.00,0.50,0.00}{##1}}}
\expandafter\def\csname PY@tok@nc\endcsname{\let\PY@bf=\textbf\def\PY@tc##1{\textcolor[rgb]{0.00,0.00,1.00}{##1}}}
\expandafter\def\csname PY@tok@ge\endcsname{\let\PY@it=\textit}
\expandafter\def\csname PY@tok@k\endcsname{\let\PY@bf=\textbf\def\PY@tc##1{\textcolor[rgb]{0.00,0.50,0.00}{##1}}}
\expandafter\def\csname PY@tok@mf\endcsname{\def\PY@tc##1{\textcolor[rgb]{0.40,0.40,0.40}{##1}}}
\expandafter\def\csname PY@tok@cs\endcsname{\let\PY@it=\textit\def\PY@tc##1{\textcolor[rgb]{0.25,0.50,0.50}{##1}}}
\expandafter\def\csname PY@tok@fm\endcsname{\def\PY@tc##1{\textcolor[rgb]{0.00,0.00,1.00}{##1}}}
\expandafter\def\csname PY@tok@nn\endcsname{\let\PY@bf=\textbf\def\PY@tc##1{\textcolor[rgb]{0.00,0.00,1.00}{##1}}}
\expandafter\def\csname PY@tok@nd\endcsname{\def\PY@tc##1{\textcolor[rgb]{0.67,0.13,1.00}{##1}}}
\expandafter\def\csname PY@tok@go\endcsname{\def\PY@tc##1{\textcolor[rgb]{0.53,0.53,0.53}{##1}}}
\expandafter\def\csname PY@tok@o\endcsname{\def\PY@tc##1{\textcolor[rgb]{0.40,0.40,0.40}{##1}}}
\expandafter\def\csname PY@tok@il\endcsname{\def\PY@tc##1{\textcolor[rgb]{0.40,0.40,0.40}{##1}}}
\expandafter\def\csname PY@tok@ow\endcsname{\let\PY@bf=\textbf\def\PY@tc##1{\textcolor[rgb]{0.67,0.13,1.00}{##1}}}
\expandafter\def\csname PY@tok@sb\endcsname{\def\PY@tc##1{\textcolor[rgb]{0.73,0.13,0.13}{##1}}}
\expandafter\def\csname PY@tok@kp\endcsname{\def\PY@tc##1{\textcolor[rgb]{0.00,0.50,0.00}{##1}}}
\expandafter\def\csname PY@tok@gd\endcsname{\def\PY@tc##1{\textcolor[rgb]{0.63,0.00,0.00}{##1}}}
\expandafter\def\csname PY@tok@sd\endcsname{\let\PY@it=\textit\def\PY@tc##1{\textcolor[rgb]{0.73,0.13,0.13}{##1}}}
\expandafter\def\csname PY@tok@kr\endcsname{\let\PY@bf=\textbf\def\PY@tc##1{\textcolor[rgb]{0.00,0.50,0.00}{##1}}}
\expandafter\def\csname PY@tok@sh\endcsname{\def\PY@tc##1{\textcolor[rgb]{0.73,0.13,0.13}{##1}}}
\expandafter\def\csname PY@tok@cp\endcsname{\def\PY@tc##1{\textcolor[rgb]{0.74,0.48,0.00}{##1}}}
\expandafter\def\csname PY@tok@c\endcsname{\let\PY@it=\textit\def\PY@tc##1{\textcolor[rgb]{0.25,0.50,0.50}{##1}}}
\expandafter\def\csname PY@tok@s1\endcsname{\def\PY@tc##1{\textcolor[rgb]{0.73,0.13,0.13}{##1}}}
\expandafter\def\csname PY@tok@c1\endcsname{\let\PY@it=\textit\def\PY@tc##1{\textcolor[rgb]{0.25,0.50,0.50}{##1}}}
\expandafter\def\csname PY@tok@ne\endcsname{\let\PY@bf=\textbf\def\PY@tc##1{\textcolor[rgb]{0.82,0.25,0.23}{##1}}}
\expandafter\def\csname PY@tok@dl\endcsname{\def\PY@tc##1{\textcolor[rgb]{0.73,0.13,0.13}{##1}}}
\expandafter\def\csname PY@tok@si\endcsname{\let\PY@bf=\textbf\def\PY@tc##1{\textcolor[rgb]{0.73,0.40,0.53}{##1}}}
\expandafter\def\csname PY@tok@ch\endcsname{\let\PY@it=\textit\def\PY@tc##1{\textcolor[rgb]{0.25,0.50,0.50}{##1}}}
\expandafter\def\csname PY@tok@s\endcsname{\def\PY@tc##1{\textcolor[rgb]{0.73,0.13,0.13}{##1}}}
\expandafter\def\csname PY@tok@sr\endcsname{\def\PY@tc##1{\textcolor[rgb]{0.73,0.40,0.53}{##1}}}
\expandafter\def\csname PY@tok@nv\endcsname{\def\PY@tc##1{\textcolor[rgb]{0.10,0.09,0.49}{##1}}}
\expandafter\def\csname PY@tok@no\endcsname{\def\PY@tc##1{\textcolor[rgb]{0.53,0.00,0.00}{##1}}}
\expandafter\def\csname PY@tok@vg\endcsname{\def\PY@tc##1{\textcolor[rgb]{0.10,0.09,0.49}{##1}}}
\expandafter\def\csname PY@tok@na\endcsname{\def\PY@tc##1{\textcolor[rgb]{0.49,0.56,0.16}{##1}}}
\expandafter\def\csname PY@tok@gt\endcsname{\def\PY@tc##1{\textcolor[rgb]{0.00,0.27,0.87}{##1}}}
\expandafter\def\csname PY@tok@cm\endcsname{\let\PY@it=\textit\def\PY@tc##1{\textcolor[rgb]{0.25,0.50,0.50}{##1}}}
\expandafter\def\csname PY@tok@mb\endcsname{\def\PY@tc##1{\textcolor[rgb]{0.40,0.40,0.40}{##1}}}
\expandafter\def\csname PY@tok@m\endcsname{\def\PY@tc##1{\textcolor[rgb]{0.40,0.40,0.40}{##1}}}
\expandafter\def\csname PY@tok@mi\endcsname{\def\PY@tc##1{\textcolor[rgb]{0.40,0.40,0.40}{##1}}}
\expandafter\def\csname PY@tok@nf\endcsname{\def\PY@tc##1{\textcolor[rgb]{0.00,0.00,1.00}{##1}}}
\expandafter\def\csname PY@tok@gh\endcsname{\let\PY@bf=\textbf\def\PY@tc##1{\textcolor[rgb]{0.00,0.00,0.50}{##1}}}
\expandafter\def\csname PY@tok@err\endcsname{\def\PY@bc##1{\setlength{\fboxsep}{0pt}\fcolorbox[rgb]{1.00,0.00,0.00}{1,1,1}{\strut ##1}}}
\expandafter\def\csname PY@tok@vc\endcsname{\def\PY@tc##1{\textcolor[rgb]{0.10,0.09,0.49}{##1}}}
\expandafter\def\csname PY@tok@w\endcsname{\def\PY@tc##1{\textcolor[rgb]{0.73,0.73,0.73}{##1}}}
\expandafter\def\csname PY@tok@gu\endcsname{\let\PY@bf=\textbf\def\PY@tc##1{\textcolor[rgb]{0.50,0.00,0.50}{##1}}}
\expandafter\def\csname PY@tok@vm\endcsname{\def\PY@tc##1{\textcolor[rgb]{0.10,0.09,0.49}{##1}}}
\expandafter\def\csname PY@tok@vi\endcsname{\def\PY@tc##1{\textcolor[rgb]{0.10,0.09,0.49}{##1}}}
\expandafter\def\csname PY@tok@kd\endcsname{\let\PY@bf=\textbf\def\PY@tc##1{\textcolor[rgb]{0.00,0.50,0.00}{##1}}}
\expandafter\def\csname PY@tok@sc\endcsname{\def\PY@tc##1{\textcolor[rgb]{0.73,0.13,0.13}{##1}}}
\expandafter\def\csname PY@tok@nt\endcsname{\let\PY@bf=\textbf\def\PY@tc##1{\textcolor[rgb]{0.00,0.50,0.00}{##1}}}
\expandafter\def\csname PY@tok@nl\endcsname{\def\PY@tc##1{\textcolor[rgb]{0.63,0.63,0.00}{##1}}}
\expandafter\def\csname PY@tok@kt\endcsname{\def\PY@tc##1{\textcolor[rgb]{0.69,0.00,0.25}{##1}}}
\expandafter\def\csname PY@tok@bp\endcsname{\def\PY@tc##1{\textcolor[rgb]{0.00,0.50,0.00}{##1}}}

\def\PYZbs{\char`\\}
\def\PYZus{\char`\_}
\def\PYZob{\char`\{}
\def\PYZcb{\char`\}}
\def\PYZca{\char`\^}
\def\PYZam{\char`\&}
\def\PYZlt{\char`\<}
\def\PYZgt{\char`\>}
\def\PYZsh{\char`\#}
\def\PYZpc{\char`\%}
\def\PYZdl{\char`\$}
\def\PYZhy{\char`\-}
\def\PYZsq{\char`\'}
\def\PYZdq{\char`\"}
\def\PYZti{\char`\~}
% for compatibility with earlier versions
\def\PYZat{@}
\def\PYZlb{[}
\def\PYZrb{]}
\makeatother


    % Exact colors from NB
    \definecolor{incolor}{rgb}{0.0, 0.0, 0.5}
    \definecolor{outcolor}{rgb}{0.545, 0.0, 0.0}



    
    % Prevent overflowing lines due to hard-to-break entities
    \sloppy 
    % Setup hyperref package
    \hypersetup{
      breaklinks=true,  % so long urls are correctly broken across lines
      colorlinks=true,
      urlcolor=urlcolor,
      linkcolor=linkcolor,
      citecolor=citecolor,
      }
    % Slightly bigger margins than the latex defaults
    
    \geometry{verbose,tmargin=0.5in,bmargin=1in,lmargin=0.5in,rmargin=0.5in}
    
    

    \begin{document}

    Załadowanie bibliotek

    \begin{Verbatim}[commandchars=\\\{\}]
{\color{incolor}In [{\color{incolor}1}]:} \PY{k+kn}{import} \PY{n+nn}{pandas} \PY{k}{as} \PY{n+nn}{pd}
        \PY{k+kn}{import} \PY{n+nn}{numpy} \PY{k}{as} \PY{n+nn}{np}
        \PY{k+kn}{import} \PY{n+nn}{seaborn} \PY{k}{as} \PY{n+nn}{sns}
        \PY{o}{\PYZpc{}}\PY{k}{matplotlib} inline
        \PY{k+kn}{from} \PY{n+nn}{matplotlib} \PY{k}{import} \PY{n}{pyplot} \PY{k}{as} \PY{n}{plt}
        
        \PY{c+c1}{\PYZsh{}Ustalenie stylu wykresów jako ggplot}
        \PY{c+c1}{\PYZsh{} plt.style.use(\PYZsq{}ggplot\PYZsq{})}
        \PY{k+kn}{from} \PY{n+nn}{sklearn} \PY{k}{import} \PY{n}{preprocessing}
        \PY{k+kn}{from} \PY{n+nn}{sklearn} \PY{k}{import} \PY{n}{tree}
        \PY{k+kn}{import} \PY{n+nn}{sklearn}\PY{n+nn}{.}\PY{n+nn}{metrics} \PY{k}{as} \PY{n+nn}{metrics}
        \PY{k+kn}{from} \PY{n+nn}{sklearn}\PY{n+nn}{.}\PY{n+nn}{model\PYZus{}selection} \PY{k}{import} \PY{n}{train\PYZus{}test\PYZus{}split}
        \PY{k+kn}{import} \PY{n+nn}{pydotplus}
        \PY{k+kn}{from} \PY{n+nn}{IPython}\PY{n+nn}{.}\PY{n+nn}{display} \PY{k}{import} \PY{n}{Image}  
\end{Verbatim}


    Wczytanie danych

    \begin{Verbatim}[commandchars=\\\{\}]
{\color{incolor}In [{\color{incolor}2}]:} \PY{c+c1}{\PYZsh{} Ustalenie ścieżki do datasetu}
        \PY{n}{filename\PYZus{}mushrooms} \PY{o}{=} \PY{l+s+s1}{\PYZsq{}}\PY{l+s+s1}{./agaricus\PYZhy{}lepiota.csv}\PY{l+s+s1}{\PYZsq{}}
        
        \PY{c+c1}{\PYZsh{} Wczytanie datasetu jako dataframe}
        \PY{n}{mushrooms\PYZus{}dataframe} \PY{o}{=} \PY{n}{pd}\PY{o}{.}\PY{n}{read\PYZus{}csv}\PY{p}{(}\PY{n}{filename\PYZus{}mushrooms}\PY{p}{,} \PY{n}{sep}\PY{o}{=}\PY{l+s+s2}{\PYZdq{}}\PY{l+s+s2}{;}\PY{l+s+s2}{\PYZdq{}}\PY{p}{)}
        
        \PY{c+c1}{\PYZsh{} Wyświetlenie dataframe}
        \PY{n}{display}\PY{p}{(}\PY{n}{mushrooms\PYZus{}dataframe}\PY{p}{)}
\end{Verbatim}


    
    
    Zbiór danych ma 8124 wiersze i 23 kolumny (pierwsza kolumna to atrybut
decyzyjny, a pozostałe 22 kolumny to atrybuty warunkowe). W celu
dalszego zbadania datasetu i weryfikacji typów danych kategorycznych w
każdej kolumnie, wypisano unikalne wartości każdej kolumny. Sprawdzono
również, czy zbiór danych zawiera brakujące wartości lub niepotrzebne
kolumny.

    \begin{Verbatim}[commandchars=\\\{\}]
{\color{incolor}In [{\color{incolor}3}]:} \PY{n+nb}{print}\PY{p}{(}\PY{l+s+s2}{\PYZdq{}}\PY{l+s+s2}{Liczba różnych wartości atrybutów wraz z ich wartościami dla każdej kolumny:}\PY{l+s+s2}{\PYZdq{}}\PY{p}{)}
        \PY{k}{for} \PY{n}{x} \PY{o+ow}{in} \PY{n}{mushrooms\PYZus{}dataframe}\PY{o}{.}\PY{n}{columns}\PY{p}{:}
            \PY{n}{x\PYZus{}unique} \PY{o}{=} \PY{n}{mushrooms\PYZus{}dataframe}\PY{p}{[}\PY{n}{x}\PY{p}{]}\PY{o}{.}\PY{n}{unique}\PY{p}{(}\PY{p}{)}
            \PY{n+nb}{print}\PY{p}{(}\PY{l+s+s2}{\PYZdq{}}\PY{l+s+si}{\PYZob{}:\PYZgt{}25\PYZcb{}}\PY{l+s+s2}{: }\PY{l+s+si}{\PYZob{}:\PYZgt{}2\PYZcb{}}\PY{l+s+s2}{ }\PY{l+s+si}{\PYZob{}\PYZcb{}}\PY{l+s+s2}{\PYZdq{}}\PY{o}{.}\PY{n}{format}\PY{p}{(}\PY{n}{x}\PY{p}{,} \PY{n}{x\PYZus{}unique}\PY{o}{.}\PY{n}{shape}\PY{p}{[}\PY{l+m+mi}{0}\PY{p}{]}\PY{p}{,} \PY{n}{x\PYZus{}unique}\PY{p}{)}\PY{p}{)}
\end{Verbatim}


    \begin{Verbatim}[commandchars=\\\{\}]
Liczba różnych wartości atrybutów wraz z ich wartościami dla każdej kolumny:
                  classes:  2 ['p' 'e']
                cap-shape:  6 ['x' 'b' 's' 'f' 'k' 'c']
              cap-surface:  4 ['s' 'y' 'f' 'g']
                cap-color: 10 ['n' 'y' 'w' 'g' 'e' 'p' 'b' 'u' 'c' 'r']
                  bruises:  2 ['t' 'f']
                     odor:  9 ['p' 'a' 'l' 'n' 'f' 'c' 'y' 's' 'm']
          gill-attachment:  2 ['f' 'a']
             gill-spacing:  2 ['c' 'w']
                gill-size:  2 ['n' 'b']
               gill-color: 12 ['k' 'n' 'g' 'p' 'w' 'h' 'u' 'e' 'b' 'r' 'y' 'o']
              stalk-shape:  2 ['e' 't']
               stalk-root:  5 ['e' 'c' 'b' 'r' '?']
 stalk-surface-above-ring:  4 ['s' 'f' 'k' 'y']
 stalk-surface-below-ring:  4 ['s' 'f' 'y' 'k']
   stalk-color-above-ring:  9 ['w' 'g' 'p' 'n' 'b' 'e' 'o' 'c' 'y']
   stalk-color-below-ring:  9 ['w' 'p' 'g' 'b' 'n' 'e' 'y' 'o' 'c']
                veil-type:  1 ['p']
               veil-color:  4 ['w' 'n' 'o' 'y']
              ring-number:  3 ['o' 't' 'n']
                ring-type:  5 ['p' 'e' 'l' 'f' 'n']
        spore-print-color:  9 ['k' 'n' 'u' 'h' 'w' 'r' 'o' 'y' 'b']
               population:  6 ['s' 'n' 'a' 'v' 'y' 'c']
                  habitat:  7 ['u' 'g' 'm' 'd' 'p' 'w' 'l']

    \end{Verbatim}

    Zauważono, że spośród 22 atrybutów warunkowych, jedynie `veil-type'
zawiera tylko jedną wartość ``p''. Zatem atrybut ten nie zapewnia żadnej
wartości dodanej do klasyfikatora. Podjęto decyzję o usunięciu tej
kolumny - utworzono generyczny kod usuwający wszystkie kolumny
zawierające jedną wartość.

    \begin{Verbatim}[commandchars=\\\{\}]
{\color{incolor}In [{\color{incolor}4}]:} \PY{n+nb}{print}\PY{p}{(}\PY{l+s+s2}{\PYZdq{}}\PY{l+s+s2}{Rozmiar mushrooms\PYZus{}dataframe przed usunięciem: }\PY{l+s+s2}{\PYZdq{}}\PY{p}{,}\PY{n}{mushrooms\PYZus{}dataframe}\PY{o}{.}\PY{n}{shape}\PY{p}{)}     
        
        \PY{c+c1}{\PYZsh{} Usunięcie kolumn zawierających jedną wartośc}
        \PY{k}{for} \PY{n}{col} \PY{o+ow}{in} \PY{n}{mushrooms\PYZus{}dataframe}\PY{o}{.}\PY{n}{columns}\PY{o}{.}\PY{n}{values}\PY{p}{:}
            \PY{n}{col\PYZus{}unique} \PY{o}{=} \PY{n}{mushrooms\PYZus{}dataframe}\PY{p}{[}\PY{n}{col}\PY{p}{]}\PY{o}{.}\PY{n}{unique}\PY{p}{(}\PY{p}{)}
            \PY{k}{if} \PY{n+nb}{len}\PY{p}{(}\PY{n}{col\PYZus{}unique}\PY{p}{)} \PY{o}{==} \PY{l+m+mi}{1}\PY{p}{:}
                \PY{n+nb}{print}\PY{p}{(}\PY{l+s+s2}{\PYZdq{}}\PY{l+s+s2}{Usunięto kolumnę }\PY{l+s+s2}{\PYZsq{}}\PY{l+s+si}{\PYZob{}\PYZcb{}}\PY{l+s+s2}{\PYZsq{}}\PY{l+s+s2}{,która zawiera tylko jedną wartość: }\PY{l+s+si}{\PYZob{}\PYZcb{}}\PY{l+s+s2}{\PYZdq{}}\PY{o}{.}\PY{n}{format}\PY{p}{(}\PY{n}{col}\PY{p}{,} \PY{n}{col\PYZus{}unique}\PY{p}{[}\PY{l+m+mi}{0}\PY{p}{]}\PY{p}{)}\PY{p}{)}
                \PY{n}{mushrooms\PYZus{}dataframe} \PY{o}{=} \PY{n}{mushrooms\PYZus{}dataframe}\PY{o}{.}\PY{n}{drop}\PY{p}{(}\PY{n}{col}\PY{p}{,} \PY{l+m+mi}{1}\PY{p}{)}
                
        \PY{n+nb}{print}\PY{p}{(}\PY{l+s+s2}{\PYZdq{}}\PY{l+s+s2}{Rozmiar mushrooms\PYZus{}dataframe po usunięciu: }\PY{l+s+s2}{\PYZdq{}}\PY{p}{,}\PY{n}{mushrooms\PYZus{}dataframe}\PY{o}{.}\PY{n}{shape}\PY{p}{)}
\end{Verbatim}


    \begin{Verbatim}[commandchars=\\\{\}]
Rozmiar mushrooms\_dataframe przed usunięciem:  (8124, 23)
Usunięto kolumnę 'veil-type',która zawiera tylko jedną wartość: p
Rozmiar mushrooms\_dataframe po usunięciu:  (8124, 22)

    \end{Verbatim}

    Stwierdzono również, że kolumna `stalk-root' zawiera brakujące wartości.
Zbadano udział brakujących wartości w zbiorze - utworzono generyczny kod
badający udziały brakująych wartości.

    \begin{Verbatim}[commandchars=\\\{\}]
{\color{incolor}In [{\color{incolor}5}]:} \PY{k}{for} \PY{n}{x} \PY{o+ow}{in} \PY{n}{mushrooms\PYZus{}dataframe}\PY{o}{.}\PY{n}{columns}\PY{p}{:}
            \PY{n}{x\PYZus{}unique} \PY{o}{=} \PY{n}{mushrooms\PYZus{}dataframe}\PY{p}{[}\PY{n}{x}\PY{p}{]}\PY{o}{.}\PY{n}{unique}\PY{p}{(}\PY{p}{)}
            \PY{k}{if} \PY{l+s+s1}{\PYZsq{}}\PY{l+s+s1}{?}\PY{l+s+s1}{\PYZsq{}} \PY{o+ow}{in} \PY{n}{x\PYZus{}unique}\PY{p}{:}
                \PY{n}{column} \PY{o}{=} \PY{n}{mushrooms\PYZus{}dataframe}\PY{p}{[}\PY{n}{x}\PY{p}{]}
                \PY{n}{column\PYZus{}count} \PY{o}{=} \PY{n}{column}\PY{o}{.}\PY{n}{count}\PY{p}{(}\PY{p}{)}
                \PY{n}{column\PYZus{}value\PYZus{}count} \PY{o}{=} \PY{n}{column}\PY{o}{.}\PY{n}{value\PYZus{}counts}\PY{p}{(}\PY{p}{)}
        
                \PY{n+nb}{print}\PY{p}{(}\PY{l+s+s2}{\PYZdq{}}\PY{l+s+s2}{Liczba obiektów w zależności od kategorii i ich udział procentowy dla klasy }\PY{l+s+s2}{\PYZsq{}}\PY{l+s+si}{\PYZob{}\PYZcb{}}\PY{l+s+s2}{\PYZsq{}}\PY{l+s+s2}{:}\PY{l+s+se}{\PYZbs{}n}\PY{l+s+s2}{\PYZdq{}}\PY{o}{.}\PY{n}{format}\PY{p}{(}\PY{n}{x}\PY{p}{)}\PY{p}{)}
                \PY{n}{stat} \PY{o}{=} \PY{n}{column\PYZus{}value\PYZus{}count}\PY{o}{.}\PY{n}{to\PYZus{}frame}\PY{p}{(}\PY{p}{)}
                \PY{n}{stat}\PY{p}{[}\PY{l+s+s1}{\PYZsq{}}\PY{l+s+s1}{percent}\PY{l+s+s1}{\PYZsq{}}\PY{p}{]} \PY{o}{=} \PY{l+m+mf}{100.} \PY{o}{*} \PY{n}{column\PYZus{}value\PYZus{}count} \PY{o}{/} \PY{n}{column\PYZus{}count}
                \PY{n+nb}{print}\PY{p}{(}\PY{n}{stat}\PY{p}{)}
        
                \PY{n}{fig} \PY{o}{=} \PY{n}{plt}\PY{o}{.}\PY{n}{figure}\PY{p}{(}\PY{p}{)}
                \PY{n}{fig}\PY{o}{.}\PY{n}{patch}\PY{o}{.}\PY{n}{set\PYZus{}facecolor}\PY{p}{(}\PY{l+s+s1}{\PYZsq{}}\PY{l+s+s1}{xkcd:white}\PY{l+s+s1}{\PYZsq{}}\PY{p}{)}
                \PY{n}{ax} \PY{o}{=} \PY{n}{sns}\PY{o}{.}\PY{n}{countplot}\PY{p}{(}\PY{n}{x}\PY{o}{=}\PY{n}{x}\PY{p}{,} \PY{n}{data}\PY{o}{=}\PY{n}{mushrooms\PYZus{}dataframe}\PY{p}{)}
                \PY{n}{ax}\PY{o}{.}\PY{n}{set\PYZus{}title}\PY{p}{(}\PY{l+s+s2}{\PYZdq{}}\PY{l+s+s2}{Liczba obiektów w zależności od kategorii dla }\PY{l+s+s2}{\PYZsq{}}\PY{l+s+si}{\PYZob{}\PYZcb{}}\PY{l+s+s2}{\PYZsq{}}\PY{l+s+s2}{\PYZdq{}}\PY{o}{.}\PY{n}{format}\PY{p}{(}\PY{n}{x}\PY{p}{)}\PY{p}{)}
                
        \PY{c+c1}{\PYZsh{}         for p in ax.patches:}
        \PY{c+c1}{\PYZsh{}             height = p.get\PYZus{}height()}
        \PY{c+c1}{\PYZsh{}             ax.text(p.get\PYZus{}x()+0.25, height+ 3, \PYZsq{}n=\PYZpc{}.0f\PYZsq{}\PYZpc{}(height))}
                
                \PY{k}{for} \PY{n}{p} \PY{o+ow}{in} \PY{n}{ax}\PY{o}{.}\PY{n}{patches}\PY{p}{:}
                    \PY{n}{ax}\PY{o}{.}\PY{n}{annotate}\PY{p}{(}\PY{l+s+s1}{\PYZsq{}}\PY{l+s+si}{\PYZob{}:.0f\PYZcb{}}\PY{l+s+se}{\PYZbs{}n}\PY{l+s+s1}{(}\PY{l+s+si}{\PYZob{}:.2f\PYZcb{}}\PY{l+s+s1}{\PYZpc{}}\PY{l+s+s1}{)}\PY{l+s+s1}{\PYZsq{}}\PY{o}{.}\PY{n}{format}\PY{p}{(}\PY{n}{p}\PY{o}{.}\PY{n}{get\PYZus{}height}\PY{p}{(}\PY{p}{)}\PY{p}{,}
                                                           \PY{l+m+mf}{100.} \PY{o}{*} \PY{n}{p}\PY{o}{.}\PY{n}{get\PYZus{}height}\PY{p}{(}\PY{p}{)} \PY{o}{/} \PY{n}{column\PYZus{}count}\PY{p}{)}\PY{p}{,}
                                                           \PY{p}{(}\PY{n}{p}\PY{o}{.}\PY{n}{get\PYZus{}x}\PY{p}{(}\PY{p}{)}\PY{o}{+}\PY{l+m+mf}{0.05}\PY{p}{,} \PY{n}{p}\PY{o}{.}\PY{n}{get\PYZus{}height}\PY{p}{(}\PY{p}{)}\PY{o}{/}\PY{o}{/}\PY{l+m+mi}{2}\PY{p}{)}\PY{p}{)}
                \PY{n}{plt}\PY{o}{.}\PY{n}{show}\PY{p}{(}\PY{p}{)}
\end{Verbatim}


    \begin{Verbatim}[commandchars=\\\{\}]
Liczba obiektów w zależności od kategorii i ich udział procentowy dla klasy 'stalk-root':

   stalk-root    percent
b        3776  46.479567
?        2480  30.526834
e        1120  13.786312
c         556   6.843919
r         192   2.363368

    \end{Verbatim}

    \begin{center}
    \adjustimage{max size={0.9\linewidth}{0.9\paperheight}}{output_9_1.png}
    \end{center}
    { \hspace*{\fill} \\}
    
    Możliwe działania do podjęcia w pzypadku występowania brakujących danych
to m.in. usunięcie kolumn lub wierszy zawierających brakujące dane,
wypełninie brakujących wartości inną wartościa np. z poprzedniej lub
następnej komórki. Stwierdzono, że udzial procentowy brakujących
wartości (`?') dla atrybutu `stalk-root' wynosi ponad 30,5\%. Podjeto
decyzję sporządzeniu dwóch wersji zbioru danych: z usuniętymi wierszami
oraz z usuniętymi kolumnami zawierającymi brakujące wartości.\par
\vskip 0.2in
Utworzono generyczny kod oczyszczający zbiór danych z wierszy
zawierających brakujące wartości:

    \begin{Verbatim}[commandchars=\\\{\}]
{\color{incolor}In [{\color{incolor}6}]:} \PY{c+c1}{\PYZsh{} Wykonanie kopii danych}
        \PY{n}{mushrooms\PYZus{}dataframe\PYZus{}dropped\PYZus{}rows} \PY{o}{=} \PY{n}{mushrooms\PYZus{}dataframe}\PY{o}{.}\PY{n}{copy}\PY{p}{(}\PY{n}{deep}\PY{o}{=}\PY{k+kc}{True}\PY{p}{)}
        
        \PY{c+c1}{\PYZsh{} Usunięcie wierszy}
        \PY{k}{for} \PY{n}{x} \PY{o+ow}{in} \PY{n}{mushrooms\PYZus{}dataframe\PYZus{}dropped\PYZus{}rows}\PY{o}{.}\PY{n}{columns}\PY{p}{:}
            \PY{n}{to\PYZus{}delete\PYZus{}count} \PY{o}{=} \PY{n}{mushrooms\PYZus{}dataframe\PYZus{}dropped\PYZus{}rows}\PY{p}{[}\PY{n}{mushrooms\PYZus{}dataframe\PYZus{}dropped\PYZus{}rows}\PY{p}{[}\PY{n}{x}\PY{p}{]} \PY{o}{==} \PY{l+s+s1}{\PYZsq{}}\PY{l+s+s1}{?}\PY{l+s+s1}{\PYZsq{}}\PY{p}{]}\PY{o}{.}\PY{n}{shape}\PY{p}{[}\PY{l+m+mi}{0}\PY{p}{]}
            \PY{k}{if} \PY{n}{to\PYZus{}delete\PYZus{}count} \PY{o}{\PYZgt{}} \PY{l+m+mi}{0}\PY{p}{:}
                \PY{n}{mushrooms\PYZus{}dataframe\PYZus{}dropped\PYZus{}rows} \PY{o}{=} \PY{n}{mushrooms\PYZus{}dataframe\PYZus{}dropped\PYZus{}rows}\PY{p}{[}\PY{n}{mushrooms\PYZus{}dataframe\PYZus{}dropped\PYZus{}rows}\PY{p}{[}\PY{n}{x}\PY{p}{]} \PY{o}{!=} \PY{l+s+s1}{\PYZsq{}}\PY{l+s+s1}{?}\PY{l+s+s1}{\PYZsq{}}\PY{p}{]}
                \PY{n+nb}{print}\PY{p}{(}\PY{l+s+s2}{\PYZdq{}}\PY{l+s+s2}{W kolumnie }\PY{l+s+s2}{\PYZsq{}}\PY{l+s+si}{\PYZob{}\PYZcb{}}\PY{l+s+s2}{\PYZsq{}}\PY{l+s+s2}{ usunięto }\PY{l+s+si}{\PYZob{}\PYZcb{}}\PY{l+s+s2}{ wierszy zawierających brakujące wartości.}\PY{l+s+s2}{\PYZdq{}}\PY{o}{.}\PY{n}{format}\PY{p}{(}\PY{n}{x}\PY{p}{,} \PY{n}{to\PYZus{}delete\PYZus{}count}\PY{p}{)}\PY{p}{)}
        
        \PY{n+nb}{print}\PY{p}{(}\PY{l+s+s2}{\PYZdq{}}\PY{l+s+s2}{mushrooms\PYZus{}dataframe\PYZus{}dropped\PYZus{}rows: }\PY{l+s+s2}{\PYZdq{}}\PY{p}{,}\PY{n}{mushrooms\PYZus{}dataframe\PYZus{}dropped\PYZus{}rows}\PY{o}{.}\PY{n}{shape}\PY{p}{)}
        
        \PY{n+nb}{print}\PY{p}{(}\PY{l+s+s2}{\PYZdq{}}\PY{l+s+se}{\PYZbs{}n}\PY{l+s+s2}{ Podział atrybutu decyzyjnego:}\PY{l+s+s2}{\PYZdq{}}\PY{p}{)}
        \PY{n+nb}{print}\PY{p}{(}\PY{n}{mushrooms\PYZus{}dataframe\PYZus{}dropped\PYZus{}rows}\PY{p}{[}\PY{l+s+s1}{\PYZsq{}}\PY{l+s+s1}{classes}\PY{l+s+s1}{\PYZsq{}}\PY{p}{]}\PY{o}{.}\PY{n}{value\PYZus{}counts}\PY{p}{(}\PY{p}{)}\PY{p}{)}
\end{Verbatim}


    \begin{Verbatim}[commandchars=\\\{\}]
W kolumnie 'stalk-root' usunięto 2480 wierszy zawierających brakujące wartości.
mushrooms\_dataframe\_dropped\_rows:  (5644, 22)

 Podział atrybutu decyzyjnego:
e    3488
p    2156
Name: classes, dtype: int64

    \end{Verbatim}

    Utworzono generyczny kod oczyszczający zbiór danych z kolumn
zawierających brakujące wartości powyżej zadanego progu procentowego
(25\%):

    \begin{Verbatim}[commandchars=\\\{\}]
{\color{incolor}In [{\color{incolor}7}]:} \PY{c+c1}{\PYZsh{} Próg procentowy usuwania kolumn z brakującymi wartościami}
        \PY{n}{drop\PYZus{}percentage} \PY{o}{=} \PY{l+m+mf}{0.25}
        
        \PY{c+c1}{\PYZsh{} Wykonanie kopii danych}
        \PY{n}{mushrooms\PYZus{}dataframe\PYZus{}dropped\PYZus{}cols} \PY{o}{=} \PY{n}{mushrooms\PYZus{}dataframe}\PY{o}{.}\PY{n}{copy}\PY{p}{(}\PY{n}{deep}\PY{o}{=}\PY{k+kc}{True}\PY{p}{)}
        
        \PY{c+c1}{\PYZsh{} Zastąpienie znaku ? wartością nan}
        \PY{k}{for} \PY{n}{col} \PY{o+ow}{in} \PY{n}{mushrooms\PYZus{}dataframe\PYZus{}dropped\PYZus{}cols}\PY{p}{:}
            \PY{n}{mushrooms\PYZus{}dataframe\PYZus{}dropped\PYZus{}cols}\PY{o}{.}\PY{n}{loc}\PY{p}{[}\PY{n}{mushrooms\PYZus{}dataframe\PYZus{}dropped\PYZus{}cols}\PY{p}{[}\PY{n}{col}\PY{p}{]} \PY{o}{==} \PY{l+s+s1}{\PYZsq{}}\PY{l+s+s1}{?}\PY{l+s+s1}{\PYZsq{}}\PY{p}{,} \PY{n}{col}\PY{p}{]} \PY{o}{=} \PY{n}{np}\PY{o}{.}\PY{n}{nan}
        
        \PY{c+c1}{\PYZsh{} Usunięcie kolumn}
        \PY{k}{for} \PY{n}{col} \PY{o+ow}{in} \PY{n}{mushrooms\PYZus{}dataframe\PYZus{}dropped\PYZus{}cols}\PY{o}{.}\PY{n}{columns}\PY{o}{.}\PY{n}{values}\PY{p}{:}
            \PY{n}{no\PYZus{}rows} \PY{o}{=} \PY{n}{mushrooms\PYZus{}dataframe\PYZus{}dropped\PYZus{}cols}\PY{p}{[}\PY{n}{col}\PY{p}{]}\PY{o}{.}\PY{n}{isnull}\PY{p}{(}\PY{p}{)}\PY{o}{.}\PY{n}{sum}\PY{p}{(}\PY{p}{)}
            \PY{n}{percentage} \PY{o}{=} \PY{n}{no\PYZus{}rows} \PY{o}{/} \PY{n}{mushrooms\PYZus{}dataframe\PYZus{}dropped\PYZus{}cols}\PY{o}{.}\PY{n}{shape}\PY{p}{[}\PY{l+m+mi}{0}\PY{p}{]}
            \PY{k}{if} \PY{n}{percentage} \PY{o}{\PYZgt{}}\PY{o}{=} \PY{n}{drop\PYZus{}percentage}\PY{p}{:}
                \PY{k}{del} \PY{n}{mushrooms\PYZus{}dataframe\PYZus{}dropped\PYZus{}cols}\PY{p}{[}\PY{n}{col}\PY{p}{]}
                \PY{n+nb}{print}\PY{p}{(}\PY{l+s+s2}{\PYZdq{}}\PY{l+s+s2}{Kolumna }\PY{l+s+s2}{\PYZsq{}}\PY{l+s+si}{\PYZob{}\PYZcb{}}\PY{l+s+s2}{\PYZsq{}}\PY{l+s+s2}{ zawierająca }\PY{l+s+si}{\PYZob{}\PYZcb{}}\PY{l+s+s2}{ brakujących wartości (}\PY{l+s+si}{\PYZob{}\PYZcb{}}\PY{l+s+s2}{\PYZpc{}}\PY{l+s+s2}{ procent zbioru) została usunięta.}\PY{l+s+s2}{\PYZdq{}}\PY{o}{.}\PY{n}{format}\PY{p}{(}\PY{n}{col}\PY{p}{,} \PY{n}{no\PYZus{}rows}\PY{p}{,} \PY{n}{percentage}\PY{p}{)}\PY{p}{)}
                
        \PY{n+nb}{print}\PY{p}{(}\PY{l+s+s2}{\PYZdq{}}\PY{l+s+s2}{mushrooms\PYZus{}dataframe\PYZus{}dropped\PYZus{}cols: }\PY{l+s+s2}{\PYZdq{}}\PY{p}{,}\PY{n}{mushrooms\PYZus{}dataframe\PYZus{}dropped\PYZus{}rows}\PY{o}{.}\PY{n}{shape}\PY{p}{)}
        
        \PY{n+nb}{print}\PY{p}{(}\PY{l+s+s2}{\PYZdq{}}\PY{l+s+se}{\PYZbs{}n}\PY{l+s+s2}{ Podział atrybutu decyzyjnego:}\PY{l+s+s2}{\PYZdq{}}\PY{p}{)}
        \PY{n+nb}{print}\PY{p}{(}\PY{n}{mushrooms\PYZus{}dataframe\PYZus{}dropped\PYZus{}cols}\PY{p}{[}\PY{l+s+s1}{\PYZsq{}}\PY{l+s+s1}{classes}\PY{l+s+s1}{\PYZsq{}}\PY{p}{]}\PY{o}{.}\PY{n}{value\PYZus{}counts}\PY{p}{(}\PY{p}{)}\PY{p}{)}
\end{Verbatim}


    \begin{Verbatim}[commandchars=\\\{\}]
Kolumna 'stalk-root' zawierająca 2480 brakujących wartości (0.3052683407188577\% procent zbioru) została usunięta.
mushrooms\_dataframe\_dropped\_cols:  (5644, 22)

 Podział atrybutu decyzyjnego:
e    4208
p    3916
Name: classes, dtype: int64

    \end{Verbatim}

    Przed przystąpieniem do budowy drzewa decyzyjnego należy zakodować
wartości atrybutów (kolumn). Do zakodowania wartości kategorycznych
użyta zostanie technika kodowania etykiet, która konwertuje każdą
wartość w kolumnie na liczbę.

    \begin{Verbatim}[commandchars=\\\{\}]
{\color{incolor}In [{\color{incolor}8}]:} \PY{n+nb}{print}\PY{p}{(}\PY{l+s+s1}{\PYZsq{}}\PY{l+s+s1}{Kodowanie danych z usuniętymi wierszami:}\PY{l+s+s1}{\PYZsq{}}\PY{p}{)}
        \PY{n}{le\PYZus{}rows} \PY{o}{=} \PY{n}{preprocessing}\PY{o}{.}\PY{n}{LabelEncoder}\PY{p}{(}\PY{p}{)}
        \PY{k}{for} \PY{n}{column} \PY{o+ow}{in} \PY{n}{mushrooms\PYZus{}dataframe\PYZus{}dropped\PYZus{}rows}\PY{o}{.}\PY{n}{columns}\PY{p}{:}
            \PY{n}{mushrooms\PYZus{}dataframe\PYZus{}dropped\PYZus{}rows}\PY{p}{[}\PY{n}{column}\PY{p}{]} \PY{o}{=} \PY{n}{le\PYZus{}rows}\PY{o}{.}\PY{n}{fit\PYZus{}transform}\PY{p}{(}\PY{n}{mushrooms\PYZus{}dataframe\PYZus{}dropped\PYZus{}rows}\PY{p}{[}\PY{n}{column}\PY{p}{]}\PY{p}{)}
        
        \PY{n+nb}{print}\PY{p}{(}\PY{l+s+s2}{\PYZdq{}}\PY{l+s+s2}{Liczba różnych wartości atrybutów wraz z ich wartościami dla każdej kolumny po zakodowaniu:}\PY{l+s+s2}{\PYZdq{}}\PY{p}{)}
        \PY{k}{for} \PY{n}{x} \PY{o+ow}{in} \PY{n}{mushrooms\PYZus{}dataframe\PYZus{}dropped\PYZus{}rows}\PY{o}{.}\PY{n}{columns}\PY{p}{:}
            \PY{n}{x\PYZus{}unique} \PY{o}{=} \PY{n}{mushrooms\PYZus{}dataframe\PYZus{}dropped\PYZus{}rows}\PY{p}{[}\PY{n}{x}\PY{p}{]}\PY{o}{.}\PY{n}{unique}\PY{p}{(}\PY{p}{)}
            \PY{n+nb}{print}\PY{p}{(}\PY{l+s+s2}{\PYZdq{}}\PY{l+s+si}{\PYZob{}:\PYZgt{}25\PYZcb{}}\PY{l+s+s2}{: }\PY{l+s+si}{\PYZob{}:\PYZgt{}2\PYZcb{}}\PY{l+s+s2}{ }\PY{l+s+si}{\PYZob{}\PYZcb{}}\PY{l+s+s2}{\PYZdq{}}\PY{o}{.}\PY{n}{format}\PY{p}{(}\PY{n}{x}\PY{p}{,} \PY{n}{x\PYZus{}unique}\PY{o}{.}\PY{n}{shape}\PY{p}{[}\PY{l+m+mi}{0}\PY{p}{]}\PY{p}{,} \PY{n}{x\PYZus{}unique}\PY{p}{)}\PY{p}{)}
        
        \PY{n+nb}{print}\PY{p}{(}\PY{l+s+s1}{\PYZsq{}}\PY{l+s+se}{\PYZbs{}n}\PY{l+s+se}{\PYZbs{}n}\PY{l+s+s1}{Kodowanie danych z usuniętymi kolumnami:}\PY{l+s+s1}{\PYZsq{}}\PY{p}{)}    
        \PY{n}{le\PYZus{}cols} \PY{o}{=} \PY{n}{preprocessing}\PY{o}{.}\PY{n}{LabelEncoder}\PY{p}{(}\PY{p}{)}
        \PY{k}{for} \PY{n}{column} \PY{o+ow}{in} \PY{n}{mushrooms\PYZus{}dataframe\PYZus{}dropped\PYZus{}cols}\PY{o}{.}\PY{n}{columns}\PY{p}{:}
            \PY{n}{mushrooms\PYZus{}dataframe\PYZus{}dropped\PYZus{}cols}\PY{p}{[}\PY{n}{column}\PY{p}{]} \PY{o}{=} \PY{n}{le\PYZus{}cols}\PY{o}{.}\PY{n}{fit\PYZus{}transform}\PY{p}{(}\PY{n}{mushrooms\PYZus{}dataframe\PYZus{}dropped\PYZus{}cols}\PY{p}{[}\PY{n}{column}\PY{p}{]}\PY{p}{)}
        
        \PY{n+nb}{print}\PY{p}{(}\PY{l+s+s2}{\PYZdq{}}\PY{l+s+s2}{Liczba różnych wartości atrybutów wraz z ich wartościami dla każdej kolumny po zakodowaniu:}\PY{l+s+s2}{\PYZdq{}}\PY{p}{)}
        \PY{k}{for} \PY{n}{x} \PY{o+ow}{in} \PY{n}{mushrooms\PYZus{}dataframe\PYZus{}dropped\PYZus{}cols}\PY{o}{.}\PY{n}{columns}\PY{p}{:}
            \PY{n}{x\PYZus{}unique} \PY{o}{=} \PY{n}{mushrooms\PYZus{}dataframe\PYZus{}dropped\PYZus{}cols}\PY{p}{[}\PY{n}{x}\PY{p}{]}\PY{o}{.}\PY{n}{unique}\PY{p}{(}\PY{p}{)}
            \PY{n+nb}{print}\PY{p}{(}\PY{l+s+s2}{\PYZdq{}}\PY{l+s+si}{\PYZob{}:\PYZgt{}25\PYZcb{}}\PY{l+s+s2}{: }\PY{l+s+si}{\PYZob{}:\PYZgt{}2\PYZcb{}}\PY{l+s+s2}{ }\PY{l+s+si}{\PYZob{}\PYZcb{}}\PY{l+s+s2}{\PYZdq{}}\PY{o}{.}\PY{n}{format}\PY{p}{(}\PY{n}{x}\PY{p}{,} \PY{n}{x\PYZus{}unique}\PY{o}{.}\PY{n}{shape}\PY{p}{[}\PY{l+m+mi}{0}\PY{p}{]}\PY{p}{,} \PY{n}{x\PYZus{}unique}\PY{p}{)}\PY{p}{)}
\end{Verbatim}


    \begin{Verbatim}[commandchars=\\\{\}]
Kodowanie danych z usuniętymi wierszami:
Liczba różnych wartości atrybutów wraz z ich wartościami dla każdej kolumny po zakodowaniu:
                  classes:  2 [1 0]
                cap-shape:  6 [5 0 4 2 3 1]
              cap-surface:  4 [2 3 0 1]
                cap-color:  8 [4 7 6 3 2 5 0 1]
                  bruises:  2 [1 0]
                     odor:  7 [6 0 3 5 2 1 4]
          gill-attachment:  2 [1 0]
             gill-spacing:  2 [0 1]
                gill-size:  2 [1 0]
               gill-color:  9 [2 3 0 4 7 1 6 5 8]
              stalk-shape:  2 [0 1]
               stalk-root:  4 [2 1 0 3]
 stalk-surface-above-ring:  4 [2 0 1 3]
 stalk-surface-below-ring:  4 [2 0 3 1]
   stalk-color-above-ring:  7 [5 2 4 3 0 1 6]
   stalk-color-below-ring:  7 [5 4 2 0 3 1 6]
               veil-color:  2 [0 1]
              ring-number:  3 [1 2 0]
                ring-type:  4 [3 0 1 2]
        spore-print-color:  6 [1 2 4 0 3 5]
               population:  6 [3 2 0 4 5 1]
                  habitat:  6 [5 1 3 0 4 2]


Kodowanie danych z usuniętymi kolumnami:
Liczba różnych wartości atrybutów wraz z ich wartościami dla każdej kolumny po zakodowaniu:
                  classes:  2 [1 0]
                cap-shape:  6 [5 0 4 2 3 1]
              cap-surface:  4 [2 3 0 1]
                cap-color: 10 [4 9 8 3 2 5 0 7 1 6]
                  bruises:  2 [1 0]
                     odor:  9 [6 0 3 5 2 1 8 7 4]
          gill-attachment:  2 [1 0]
             gill-spacing:  2 [0 1]
                gill-size:  2 [1 0]
               gill-color: 12 [ 4  5  2  7 10  3  9  1  0  8 11  6]
              stalk-shape:  2 [0 1]
 stalk-surface-above-ring:  4 [2 0 1 3]
 stalk-surface-below-ring:  4 [2 0 3 1]
   stalk-color-above-ring:  9 [7 3 6 4 0 2 5 1 8]
   stalk-color-below-ring:  9 [7 6 3 0 4 2 8 5 1]
               veil-color:  4 [2 0 1 3]
              ring-number:  3 [1 2 0]
                ring-type:  5 [4 0 2 1 3]
        spore-print-color:  9 [2 3 6 1 7 5 4 8 0]
               population:  6 [3 2 0 4 5 1]
                  habitat:  7 [5 1 3 0 4 6 2]

    \end{Verbatim}

    Mająć zakodowane dane, należy dokonać ich podziału na atrybuty warunkowe
(zmienna X\_*) i decyzyjne (zmienna Y\_*).

    \begin{Verbatim}[commandchars=\\\{\}]
{\color{incolor}In [{\color{incolor}9}]:} \PY{n}{X\PYZus{}dropped\PYZus{}rows} \PY{o}{=} \PY{n}{mushrooms\PYZus{}dataframe\PYZus{}dropped\PYZus{}rows}\PY{o}{.}\PY{n}{drop}\PY{p}{(}\PY{p}{[}\PY{l+s+s1}{\PYZsq{}}\PY{l+s+s1}{classes}\PY{l+s+s1}{\PYZsq{}}\PY{p}{]}\PY{p}{,} \PY{n}{axis}\PY{o}{=}\PY{l+m+mi}{1}\PY{p}{)}
        \PY{n}{Y\PYZus{}dropped\PYZus{}rows} \PY{o}{=} \PY{n}{mushrooms\PYZus{}dataframe\PYZus{}dropped\PYZus{}rows}\PY{p}{[}\PY{l+s+s1}{\PYZsq{}}\PY{l+s+s1}{classes}\PY{l+s+s1}{\PYZsq{}}\PY{p}{]}
        
        \PY{n}{X\PYZus{}dropped\PYZus{}cols} \PY{o}{=} \PY{n}{mushrooms\PYZus{}dataframe\PYZus{}dropped\PYZus{}cols}\PY{o}{.}\PY{n}{drop}\PY{p}{(}\PY{p}{[}\PY{l+s+s1}{\PYZsq{}}\PY{l+s+s1}{classes}\PY{l+s+s1}{\PYZsq{}}\PY{p}{]}\PY{p}{,} \PY{n}{axis}\PY{o}{=}\PY{l+m+mi}{1}\PY{p}{)}
        \PY{n}{Y\PYZus{}dropped\PYZus{}cols} \PY{o}{=} \PY{n}{mushrooms\PYZus{}dataframe\PYZus{}dropped\PYZus{}cols}\PY{p}{[}\PY{l+s+s1}{\PYZsq{}}\PY{l+s+s1}{classes}\PY{l+s+s1}{\PYZsq{}}\PY{p}{]}
\end{Verbatim}


    Kolejnym podziałem, który należy wykonać, jest podział danych na część
treningową i testową. Założono, że rozmiar części testowej będzie
wynosił 33\% wszystkich danych. W celu zachowania powtarzalności wyników
parametr random\_state ustawiono na wartość 30 (ustawienie innej
wartości bedzie powodowąło wygnerowanie innego podziału danych i innego
drzewa decyzyjnego).

    \begin{Verbatim}[commandchars=\\\{\}]
{\color{incolor}In [{\color{incolor}10}]:} \PY{n}{random\PYZus{}state} \PY{o}{=} \PY{l+m+mi}{30}
         
         \PY{n}{X\PYZus{}train\PYZus{}dr}\PY{p}{,} \PY{n}{X\PYZus{}test\PYZus{}dr} \PY{p}{,}\PY{n}{Y\PYZus{}train\PYZus{}dr}\PY{p}{,} \PY{n}{Y\PYZus{}test\PYZus{}dr} \PY{o}{=} \PY{n}{train\PYZus{}test\PYZus{}split}\PY{p}{(}\PY{n}{X\PYZus{}dropped\PYZus{}rows}\PY{p}{,} \PY{n}{Y\PYZus{}dropped\PYZus{}rows}\PY{p}{,} \PY{n}{test\PYZus{}size} \PY{o}{=} \PY{l+m+mf}{0.33}\PY{p}{,} \PY{n}{random\PYZus{}state}\PY{o}{=}\PY{n}{random\PYZus{}state}\PY{p}{)}
         
         \PY{n}{X\PYZus{}train\PYZus{}dc}\PY{p}{,} \PY{n}{X\PYZus{}test\PYZus{}dc} \PY{p}{,}\PY{n}{Y\PYZus{}train\PYZus{}dc}\PY{p}{,} \PY{n}{Y\PYZus{}test\PYZus{}dc} \PY{o}{=} \PY{n}{train\PYZus{}test\PYZus{}split}\PY{p}{(}\PY{n}{X\PYZus{}dropped\PYZus{}cols}\PY{p}{,} \PY{n}{Y\PYZus{}dropped\PYZus{}cols}\PY{p}{,} \PY{n}{test\PYZus{}size} \PY{o}{=} \PY{l+m+mf}{0.33}\PY{p}{,} \PY{n}{random\PYZus{}state}\PY{o}{=}\PY{n}{random\PYZus{}state}\PY{p}{)}
\end{Verbatim}


    Utworzono funkcję sprawdzającą jakość klasyfikacji zbudowanego drzewa,
funkcję budującą drzewo oraz funkcję sprawdzająca istotność atrybutu.
Funkcji sprawdzająca jakość klasyfikacji zbudowanego drzewa, wypisuje
wartości Accuracy, czyli współczynnik dokładności modelu do danych
testowych, Precision - procent elementów będących istotnymi oraz Recall
- procent istotnych elementów, które zostały wybrane.

    \begin{Verbatim}[commandchars=\\\{\}]
{\color{incolor}In [{\color{incolor}11}]:} \PY{k}{def} \PY{n+nf}{test\PYZus{}tree}\PY{p}{(}\PY{n}{clf}\PY{p}{,} \PY{n}{X\PYZus{}train}\PY{p}{,} \PY{n}{X\PYZus{}test}\PY{p}{,} \PY{n}{Y\PYZus{}train}\PY{p}{,} \PY{n}{Y\PYZus{}test}\PY{p}{,} \PY{n}{print\PYZus{}res}\PY{o}{=}\PY{k+kc}{True}\PY{p}{)}\PY{p}{:}
             \PY{n}{clf}       \PY{o}{=} \PY{n}{clf}\PY{o}{.}\PY{n}{fit}\PY{p}{(}\PY{n}{X\PYZus{}train}\PY{p}{,} \PY{n}{Y\PYZus{}train}\PY{p}{)}
             \PY{n}{score}     \PY{o}{=} \PY{n}{clf}\PY{o}{.}\PY{n}{score}\PY{p}{(}\PY{n}{X\PYZus{}test}\PY{p}{,} \PY{n}{Y\PYZus{}test}\PY{p}{)}
             \PY{n}{precision} \PY{o}{=} \PY{n}{metrics}\PY{o}{.}\PY{n}{precision\PYZus{}score}\PY{p}{(}\PY{n}{Y\PYZus{}test}\PY{p}{,} \PY{n}{clf}\PY{o}{.}\PY{n}{predict}\PY{p}{(}\PY{n}{X\PYZus{}test}\PY{p}{)}\PY{p}{)}
             \PY{n}{recall}    \PY{o}{=} \PY{n}{metrics}\PY{o}{.}\PY{n}{recall\PYZus{}score}\PY{p}{(}\PY{n}{Y\PYZus{}test}\PY{p}{,} \PY{n}{clf}\PY{o}{.}\PY{n}{predict}\PY{p}{(}\PY{n}{X\PYZus{}test}\PY{p}{)}\PY{p}{)}
             \PY{n}{res} \PY{o}{=} \PY{p}{(}\PY{n}{score}\PY{p}{,} \PY{n}{precision}\PY{p}{,}\PY{n}{recall}\PY{p}{)}
             \PY{k}{if} \PY{n}{print\PYZus{}res}\PY{p}{:}
                 \PY{n+nb}{print}\PY{p}{(}\PY{l+s+s2}{\PYZdq{}}\PY{l+s+s2}{Accuracy = }\PY{l+s+si}{\PYZpc{}f}\PY{l+s+s2}{ / Precision = }\PY{l+s+si}{\PYZpc{}f}\PY{l+s+s2}{ / Recall = }\PY{l+s+si}{\PYZpc{}f}\PY{l+s+s2}{\PYZdq{}} \PY{o}{\PYZpc{}} \PY{n}{res}\PY{p}{)}
             \PY{k}{return} \PY{n}{res}
         
         \PY{k}{def} \PY{n+nf}{build\PYZus{}tree}\PY{p}{(}\PY{n}{X}\PY{p}{,} \PY{n}{X\PYZus{}train}\PY{p}{,} \PY{n}{X\PYZus{}test}\PY{p}{,} \PY{n}{Y\PYZus{}train}\PY{p}{,} \PY{n}{Y\PYZus{}test}\PY{p}{,} \PY{n}{random\PYZus{}state}\PY{p}{,} \PY{o}{*}\PY{o}{*}\PY{n}{kwargs}\PY{p}{)}\PY{p}{:}
             \PY{n}{clf} \PY{o}{=} \PY{n}{tree}\PY{o}{.}\PY{n}{DecisionTreeClassifier}\PY{p}{(}\PY{n}{random\PYZus{}state}\PY{o}{=}\PY{n}{random\PYZus{}state}\PY{p}{,} \PY{o}{*}\PY{o}{*}\PY{n}{kwargs}\PY{p}{)}
             \PY{n}{clf} \PY{o}{=} \PY{n}{clf}\PY{o}{.}\PY{n}{fit}\PY{p}{(}\PY{n}{X\PYZus{}train}\PY{p}{,} \PY{n}{Y\PYZus{}train}\PY{p}{)}
         
             \PY{n}{dot\PYZus{}data} \PY{o}{=} \PY{n}{tree}\PY{o}{.}\PY{n}{export\PYZus{}graphviz}\PY{p}{(}\PY{n}{clf}\PY{p}{,} \PY{n}{out\PYZus{}file}\PY{o}{=}\PY{k+kc}{None}\PY{p}{,}  
                                             \PY{n}{feature\PYZus{}names}\PY{o}{=}\PY{n}{X}\PY{o}{.}\PY{n}{columns}\PY{p}{,} 
                                             \PY{n}{class\PYZus{}names}\PY{o}{=}\PY{p}{[}\PY{l+s+s1}{\PYZsq{}}\PY{l+s+s1}{p}\PY{l+s+s1}{\PYZsq{}}\PY{p}{,}\PY{l+s+s1}{\PYZsq{}}\PY{l+s+s1}{e}\PY{l+s+s1}{\PYZsq{}}\PY{p}{]}\PY{p}{,}
                                             \PY{n}{filled}\PY{o}{=}\PY{k+kc}{True}\PY{p}{,} \PY{n}{rounded}\PY{o}{=}\PY{k+kc}{True}\PY{p}{,}  
                                             \PY{n}{special\PYZus{}characters}\PY{o}{=}\PY{k+kc}{True}\PY{p}{)}  
             \PY{n}{graph} \PY{o}{=} \PY{n}{pydotplus}\PY{o}{.}\PY{n}{graph\PYZus{}from\PYZus{}dot\PYZus{}data}\PY{p}{(}\PY{n}{dot\PYZus{}data}\PY{p}{)}  
             \PY{n}{display}\PY{p}{(}\PY{n}{Image}\PY{p}{(}\PY{n}{graph}\PY{o}{.}\PY{n}{create\PYZus{}png}\PY{p}{(}\PY{p}{)}\PY{p}{)}\PY{p}{)}
             
             \PY{n}{test\PYZus{}tree}\PY{p}{(}\PY{n}{clf}\PY{p}{,} \PY{n}{X\PYZus{}train}\PY{p}{,} \PY{n}{X\PYZus{}test}\PY{p}{,} \PY{n}{Y\PYZus{}train}\PY{p}{,} \PY{n}{Y\PYZus{}test}\PY{p}{)}\PY{p}{;}
             
             \PY{k}{return} \PY{n}{clf}
         
         \PY{k}{def} \PY{n+nf}{attribute\PYZus{}importance}\PY{p}{(}\PY{n}{clf}\PY{p}{,} \PY{n}{X}\PY{p}{)}\PY{p}{:}
             \PY{n}{attrs} \PY{o}{=} \PY{n}{X}\PY{o}{.}\PY{n}{columns}\PY{o}{.}\PY{n}{values}
             \PY{n}{attr\PYZus{}importance} \PY{o}{=} \PY{n}{clf}\PY{o}{.}\PY{n}{feature\PYZus{}importances\PYZus{}}
             \PY{n}{sorted\PYZus{}attr\PYZus{}importance} \PY{o}{=} \PY{n}{np}\PY{o}{.}\PY{n}{argsort}\PY{p}{(}\PY{n}{attr\PYZus{}importance}\PY{p}{)}
             \PY{n}{range\PYZus{}sorted\PYZus{}attr\PYZus{}importance} \PY{o}{=} \PY{n+nb}{range}\PY{p}{(}\PY{n+nb}{len}\PY{p}{(}\PY{n}{sorted\PYZus{}attr\PYZus{}importance}\PY{p}{)}\PY{p}{)}
             
             \PY{n}{plt}\PY{o}{.}\PY{n}{figure}\PY{p}{(}\PY{n}{figsize}\PY{o}{=}\PY{p}{(}\PY{l+m+mi}{8}\PY{p}{,} \PY{l+m+mi}{7}\PY{p}{)}\PY{p}{)}
             \PY{n}{plt}\PY{o}{.}\PY{n}{barh}\PY{p}{(}\PY{n}{range\PYZus{}sorted\PYZus{}attr\PYZus{}importance}\PY{p}{,} \PY{n}{attr\PYZus{}importance}\PY{p}{[}\PY{n}{sorted\PYZus{}attr\PYZus{}importance}\PY{p}{]}\PY{p}{)}
             \PY{n}{plt}\PY{o}{.}\PY{n}{yticks}\PY{p}{(}\PY{n}{range\PYZus{}sorted\PYZus{}attr\PYZus{}importance}\PY{p}{,} \PY{n}{attrs}\PY{p}{[}\PY{n}{sorted\PYZus{}attr\PYZus{}importance}\PY{p}{]}\PY{p}{)}
             \PY{n}{plt}\PY{o}{.}\PY{n}{xlabel}\PY{p}{(}\PY{l+s+s1}{\PYZsq{}}\PY{l+s+s1}{Importance}\PY{l+s+s1}{\PYZsq{}}\PY{p}{)}
             \PY{n}{plt}\PY{o}{.}\PY{n}{title}\PY{p}{(}\PY{l+s+s1}{\PYZsq{}}\PY{l+s+s1}{Attribute importances}\PY{l+s+s1}{\PYZsq{}}\PY{p}{)}
             \PY{n}{plt}\PY{o}{.}\PY{n}{draw}\PY{p}{(}\PY{p}{)}
             \PY{n}{plt}\PY{o}{.}\PY{n}{show}\PY{p}{(}\PY{p}{)}
\end{Verbatim}


    W oparciu o przygotowane dane zbudowano drzewa decyzyjne.
\vskip 0.2in
\begin{enumerate}
\def\labelenumi{\arabic{enumi}.}
\tightlist
\item
  Dokonano klasyfikacji zbioru danych z usuniętymi wierszami, które
  zawierały braki danych:
\end{enumerate}

    \begin{Verbatim}[commandchars=\\\{\}]
{\color{incolor}In [{\color{incolor}12}]:} \PY{n}{clf} \PY{o}{=} \PY{n}{build\PYZus{}tree}\PY{p}{(}\PY{n}{X\PYZus{}dropped\PYZus{}rows}\PY{p}{,} \PY{n}{X\PYZus{}train\PYZus{}dr}\PY{p}{,} \PY{n}{X\PYZus{}test\PYZus{}dr}\PY{p}{,} \PY{n}{Y\PYZus{}train\PYZus{}dr}\PY{p}{,} \PY{n}{Y\PYZus{}test\PYZus{}dr}\PY{p}{,} \PY{n}{random\PYZus{}state}\PY{p}{)}
\end{Verbatim}


    \begin{center}
    \adjustimage{max size={0.7\textwidth}}{output_23_0.png}
    \end{center}
    
    \begin{Verbatim}[commandchars=\\\{\}]
Accuracy = 1.000000 / Precision = 1.000000 / Recall = 1.000000

    \end{Verbatim}

    \begin{Verbatim}[commandchars=\\\{\}]
{\color{incolor}In [{\color{incolor}13}]:} \PY{n}{attribute\PYZus{}importance}\PY{p}{(}\PY{n}{clf}\PY{p}{,} \PY{n}{X\PYZus{}dropped\PYZus{}rows}\PY{p}{)}
\end{Verbatim}


    \begin{center}
    \adjustimage{max size={0.9\linewidth}{0.6\paperheight}}{output_24_0.png}
    \end{center}
    { \hspace*{\fill} \\}
    
    Wygenerowane drzewo decyzyjne posiada głębokość wynoszącą 5. Ponadto
zauważono, że najważnijszymi atrybutami są: spore-print-color oraz
gill-size. Mniejsze znaczenie mają atrybuty: habitat, stalk-shape oraz
cap-surface. Pozostałe atrybuty uznano za nieznaczące.

\begin{enumerate}
\def\labelenumi{\arabic{enumi}.}
\setcounter{enumi}{1}
\tightlist
\item
  Dokonano klasyfikacji zbioru danych z usuniętymi kolumnami, które
  zawierały braki danych:
\end{enumerate}

    \begin{Verbatim}[commandchars=\\\{\}]
{\color{incolor}In [{\color{incolor}14}]:} \PY{n}{clf} \PY{o}{=} \PY{n}{build\PYZus{}tree}\PY{p}{(}\PY{n}{X\PYZus{}dropped\PYZus{}cols}\PY{p}{,} \PY{n}{X\PYZus{}train\PYZus{}dc}\PY{p}{,} \PY{n}{X\PYZus{}test\PYZus{}dc} \PY{p}{,}\PY{n}{Y\PYZus{}train\PYZus{}dc}\PY{p}{,} \PY{n}{Y\PYZus{}test\PYZus{}dc}\PY{p}{,} \PY{n}{random\PYZus{}state}\PY{p}{)}
\end{Verbatim}


    \begin{center}
    \adjustimage{max size={0.9\linewidth}{0.9\paperheight}}{output_26_0.png}
    \end{center}
    
    \begin{Verbatim}[commandchars=\\\{\}]
Accuracy = 1.000000 / Precision = 1.000000 / Recall = 1.000000

    \end{Verbatim}

    \begin{Verbatim}[commandchars=\\\{\}]
{\color{incolor}In [{\color{incolor}15}]:} \PY{n}{attribute\PYZus{}importance}\PY{p}{(}\PY{n}{clf}\PY{p}{,} \PY{n}{X\PYZus{}dropped\PYZus{}cols}\PY{p}{)}
\end{Verbatim}


    \begin{center}
    \adjustimage{max size={0.9\linewidth}{0.6\paperheight}}{output_27_0.png}
    \end{center}
    
    Zauważono, że drzewo wygenerowane przy użyciu danych z usuniętymi
kolumnami jest bardziej rozbudowano niż drzewo wygenerowane przy użyciu
danych z usuniętymi wierszami, ponieważ zostało ono skonstruowano przy
użyciu zbioru o większej liczbie próbek. Wygenerowane drzewo decyzyjne
posiada głębokość wynoszącą 7. Zauważono również, że najważnijszymi
atrybutami są: gill-color, spore-print-color, population oraz gill-size.
Mniejsze znaczenie mają atrybuty: odor, stalk-shape, habitat oraz
stalk-color-above-ring. Pozostałe atrybuty uznano za nieznaczące.
\vskip 0.2in
Do kolejnych eksperymentów użyto z usuniętymi kolumnami, ponieważ
założono że większy zbiór danych może lepiej odzwierciedlać problem
klasyfikacji grzybów.
\vskip 0.2in
\begin{enumerate}
\def\labelenumi{\arabic{enumi}.}
\setcounter{enumi}{2}
\tightlist
\item
  Dokonano klasyfikacji zbioru danych z usuniętymi kolumnami ze
  zmienionymi parametrami. Parametry dobierano tak, aby zmnijeszyć
  wielkość drzewa decyzyjnego, jednocześnie zachowujać wysoką jakość
  klasyfikacji. Podczas ekspoerymentów zauważono, że w przypadku
  analizowanego zbioru dancych kryterium podziału nie ma istotnego
  wpływu na wygląd i jakość drzewa.
\end{enumerate}

    \begin{Verbatim}[commandchars=\\\{\}]
{\color{incolor}In [{\color{incolor}16}]:} \PY{n}{clf} \PY{o}{=} \PY{n}{build\PYZus{}tree}\PY{p}{(}\PY{n}{X\PYZus{}dropped\PYZus{}cols}\PY{p}{,} \PY{n}{X\PYZus{}train\PYZus{}dc}\PY{p}{,} \PY{n}{X\PYZus{}test\PYZus{}dc} \PY{p}{,}\PY{n}{Y\PYZus{}train\PYZus{}dc}\PY{p}{,} \PY{n}{Y\PYZus{}test\PYZus{}dc}\PY{p}{,} \PY{n}{random\PYZus{}state}\PY{p}{,}
                          \PY{n}{criterion}\PY{o}{=}\PY{l+s+s2}{\PYZdq{}}\PY{l+s+s2}{gini}\PY{l+s+s2}{\PYZdq{}}\PY{p}{,} \PY{n}{max\PYZus{}depth}\PY{o}{=}\PY{l+m+mi}{2}\PY{p}{,} \PY{n}{min\PYZus{}samples\PYZus{}split}\PY{o}{=}\PY{l+m+mi}{200}\PY{p}{,} \PY{n}{min\PYZus{}samples\PYZus{}leaf}\PY{o}{=}\PY{l+m+mi}{60}\PY{p}{,} \PY{n}{max\PYZus{}leaf\PYZus{}nodes}\PY{o}{=}\PY{l+m+mi}{5}\PY{p}{)}
\end{Verbatim}


    \begin{center}
    \adjustimage{max size={0.9\linewidth}{0.65\paperheight}}{output_29_0.png}
    \end{center}
    
    \begin{Verbatim}[commandchars=\\\{\}]
Accuracy = 0.941440 / Precision = 0.919575 / Recall = 0.959620

    \end{Verbatim}

    \begin{Verbatim}[commandchars=\\\{\}]
{\color{incolor}In [{\color{incolor}17}]:} \PY{n}{attribute\PYZus{}importance}\PY{p}{(}\PY{n}{clf}\PY{p}{,} \PY{n}{X\PYZus{}dropped\PYZus{}cols}\PY{p}{)}
\end{Verbatim}


    \begin{center}
    \adjustimage{max size={0.9\linewidth}{0.65\paperheight}}{output_30_0.png}
    \end{center}
    
    Ustalenie własnych parametrów drzewa decyzyjnego pozwoliło zmnijeszyć
jego głębokość do 3, przy jednoczesnym zachowaniu dobrej jakości
klasyfikacji. Współczynnik dokładności zmnijeszym się jedynie do około
94\%, wartość Precission do około 91\%, a Recall do około 95\%.
Zauważono również, że najważnijesze atrybutty, tj. gill-color,
spore-print-color, population oraz gill-size zostały zachowane.


    % Add a bibliography block to the postdoc
    
    
    
    \end{document}
